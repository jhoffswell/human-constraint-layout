%% For double-blind review submission
\documentclass[sigplan,10pt]{acmart}\settopmatter{printfolios=true}
%% For single-blind review submission
%\documentclass[sigplan,10pt,review]{acmart}\settopmatter{printfolios=true}
%% For final camera-ready submission
%\documentclass[sigplan,10pt]{acmart}\settopmatter{}

%% Note: Authors migrating a paper from traditional SIGPLAN
%% proceedings format to PACMPL format should change 'sigplan,10pt' to
%% 'acmlarge'.


%% Some recommended packages.
\usepackage{booktabs}   %% For formal tables:
                        %% http://ctan.org/pkg/booktabs
\usepackage{subcaption} %% For complex figures with subfigures/subcaptions
                        %% http://ctan.org/pkg/subcaption

\usepackage{todonotes}

%% Inline notes
\usepackage{soul}

\definecolor{red}{RGB}{178,34,34}
\definecolor{lightorange}{rgb}{1, 0.8, 0.4}

\definecolor{lightblue}{RGB}{135,206,250}
\definecolor{blue}{RGB}{30,144,255}
\definecolor{darkblue}{RGB}{0, 0, 153}

\definecolor{lightgreen}{RGB}{121, 210, 121}
\definecolor{darkgreen}{RGB}{0, 179, 0}

\definecolor{lightpurple}{RGB}{187, 153, 255}

\newcommand{\orange}[1]{\textcolor{lightorange}{\emph{#1}}}
\newcommand{\blue}[1]{\textcolor{lightblue}{#1}}
\newcommand{\green}[1]{\textcolor{lightgreen}{\emph{#1}}}
\newcommand{\purple}[1]{\textcolor{lightpurple}{\emph{#1}}}

%%%%%%%%%%%%%%%%%%%%%%%%%%%%%%%%%%%%%%%%%%%%%%%%%%%%%%%%%
% Commands to help specify inline figures in the paper. %
%%%%%%%%%%%%%%%%%%%%%%%%%%%%%%%%%%%%%%%%%%%%%%%%%%%%%%%%%
\usepackage{graphicx,calc}
\newlength\myheight
\newlength\mydepth
\settototalheight\myheight{Xygp}
\settodepth\mydepth{Xygp}
\setlength\fboxsep{0pt}
\newcommand*\inline[1]{
  \settototalheight\myheight{Xygp}
  \settodepth\mydepth{Xygp}
  \raisebox{-\mydepth}{\includegraphics[height=\myheight]{#1}}
}
\newcommand*\inlineTall[1]{
  \settototalheight\myheight{Xygp}
  \settodepth\mydepth{Xygp}
  \raisebox{-\mydepth}{\includegraphics[height=12pt]{#1}}
}


\makeatletter\if@ACM@journal\makeatother
%% Journal information (used by PACMPL format)
%% Supplied to authors by publisher for camera-ready submission
\acmJournal{PACMPL}
\acmVolume{1}
\acmNumber{1}
\acmArticle{1}
\acmYear{2017}
\acmMonth{1}
\acmDOI{10.1145/nnnnnnn.nnnnnnn}
\startPage{1}
\else\makeatother
%% Conference information (used by SIGPLAN proceedings format)
%% Supplied to authors by publisher for camera-ready submission
%\acmConference[PL'17]{ACM SIGPLAN Conference on Programming Languages}{January 01--03, 2017}{New York, NY, USA}
%\acmYear{2017}
%\acmISBN{978-x-xxxx-xxxx-x/YY/MM}
%\acmDOI{10.1145/nnnnnnn.nnnnnnn}
%\startPage{1}
%\fi


%% Copyright information
%% Supplied to authors (based on authors' rights management selection;
%% see authors.acm.org) by publisher for camera-ready submission
\setcopyright{none}             %% For review submission
%\setcopyright{acmcopyright}
%\setcopyright{acmlicensed}
%\setcopyright{rightsretained}
%\copyrightyear{2017}           %% If different from \acmYear


%% Bibliography style
\bibliographystyle{ACM-Reference-Format}
%% Citation style
%% Note: author/year citations are required for papers published as an
%% issue of PACMPL.
\citestyle{acmauthoryear}  %% For author/year citations
\citestyle{acmnumeric}     %% For numeric citations
%\setcitestyle{nosort}      %% With 'acmnumeric', to disable automatic
                            %% sorting of references within a single citation;
                            %% e.g., \cite{Smith99,Carpenter05,Baker12}
                            %% rendered as [14,5,2] rather than [2,5,14].
%\setcitesyle{nocompress}   %% With 'acmnumeric', to disable automatic
                            %% compression of sequential references within a
                            %% single citation;
                            %% e.g., \cite{Baker12,Baker14,Baker16}
                            %% rendered as [2,3,4] rather than [2-4].



\begin{document}

%% Title information
\title[Human Constraint Layout]         %% [Short Title] is optional;
{High-Level, User-Defined \\Constraints for Graph Layout}         
                                        %% when present, will be used in
                                        %% header instead of Full Title.
%\titlenote{with title note}             %% \titlenote is optional;
                                        %% can be repeated if necessary;
                                        %% contents suppressed with 'anonymous'
%\subtitle{Subtitle}                     %% \subtitle is optional
%\subtitlenote{with subtitle note}       %% \subtitlenote is optional;
                                        %% can be repeated if necessary;
                                        %% contents suppressed with 'anonymous'


%% Author information
%% Contents and number of authors suppressed with 'anonymous'.
%% Each author should be introduced by \author, followed by
%% \authornote (optional), \orcid (optional), \affiliation, and
%% \email.
%% An author may have multiple affiliations and/or emails; repeat the
%% appropriate command.
%% Many elements are not rendered, but should be provided for metadata
%% extraction tools.

%% Author with single affiliation.
\author{Jane Hoffswell}
%\authornote{with author1 note}          %% \authornote is optional;
                                        %% can be repeated if necessary
%\orcid{nnnn-nnnn-nnnn-nnnn}             %% \orcid is optional
\affiliation{
  %\position{Position1}
  %% \department is recommended
  \department{Paul G. Allen School of Computer Science \& Engineering}
  \institution{University of Washington}%% \institution is required
  %\streetaddress{Street1 Address1}
  \city{Seattle}
  \state{WA}
  \postcode{98195}
  \country{United States}
}
%\email{first1.last1@inst1.edu}          %% \email is recommended

%% Author with two affiliations and emails.
% \author{First2 Last2}
% \authornote{with author2 note}          %% \authornote is optional;
%                                         %% can be repeated if necessary
% \orcid{nnnn-nnnn-nnnn-nnnn}             %% \orcid is optional
% \affiliation{
%   \position{Position2a}
%   \department{Department2a}             %% \department is recommended
%   \institution{Institution2a}           %% \institution is required
%   \streetaddress{Street2a Address2a}
%   \city{City2a}
%   \state{State2a}
%   \postcode{Post-Code2a}
%   \country{Country2a}
% }
% \email{first2.last2@inst2a.com}         %% \email is recommended
% \affiliation{
%   \position{Position2b}
%   \department{Department2b}             %% \department is recommended
%   \institution{Institution2b}           %% \institution is required
%   \streetaddress{Street3b Address2b}
%   \city{City2b}
%   \state{State2b}
%   \postcode{Post-Code2b}
%   \country{Country2b}
% }
% \email{first2.last2@inst2b.org}         %% \email is recommended


%% Paper note
%% The \thanks command may be used to create a "paper note" ---
%% similar to a title note or an author note, but not explicitly
%% associated with a particular element.  It will appear immediately
%% above the permission/copyright statement.
% \thanks{with paper note}                %% \thanks is optional
                                        %% can be repeated if necesary
                                        %% contents suppressed with 'anonymous'

%% Abstract
%% Note: \begin{abstract}...\end{abstract} environment must come
%% before \maketitle command
%!TEX root = constraint-layout.tex
\newcommand{\paperabstract}{
	
	Constraints enable flexible graph layouts that combine the ease of automatic layouts with a design customized to the particular graph. However, constraint-based layouts often require individual constraints over node pairs and introduce a degree of separation between the constraint definition and the data domain that drives the constraint. To facilitate the specification of customized constraint layouts, we contribute \projectname, a domain specific language for specifying high-level constraints relative to properties of the backing data. Users identify nodes sets based on data or graph properties and apply high-level constraints within each set of nodes. Applying constraints to node sets minimizes specification effort on the part of the user. Furthermore, identifying sets by node properties and applying constraints to node sets facilitates reapplication of customized layouts across distinct graphs by separating the constraint generation from individual properties of the nodes. We demonstrate the simplicity and generalizability of \projectname~on a series of real-world examples from ecological networks, biological systems, and \orange{----?----}.

  \begin{classification}
  \CCScat{Human-centered computing}{}{Visualization}{Graph drawings}
  \end{classification}
}

% More formal definition of 'custmoized graph layout' - often achieved with manual customization
% Definitely need to have crisp definition
% More targeted at end-user perception

\begin{abstract}
\paperAbstract
\end{abstract}


%% 2012 ACM Computing Classification System (CSS) concepts
%% Generate at 'http://dl.acm.org/ccs/ccs.cfm'.
\begin{CCSXML}
<ccs2012>
<concept>
<concept_id>10011007.10011006.10011008</concept_id>
<concept_desc>Software and its engineering~General programming languages</concept_desc>
<concept_significance>500</concept_significance>
</concept>
<concept>
<concept_id>10003456.10003457.10003521.10003525</concept_id>
<concept_desc>Social and professional topics~History of programming languages</concept_desc>
<concept_significance>300</concept_significance>
</concept>
</ccs2012>
\end{CCSXML}

\ccsdesc[500]{Software and its engineering~General programming languages}
\ccsdesc[300]{Social and professional topics~History of programming languages}
%% End of generated code


%% Keywords
%% comma separated list
\keywords{keyword1, keyword2, keyword3}  %% \keywords is optional


%% \maketitle
%% Note: \maketitle command must come after title commands, author
%% commands, abstract environment, Computing Classification System
%% environment and commands, and keywords command.
\maketitle

% Styling
%\hyphenpenalty=10000 %% TODO: I added this, but it isn't playing well with this layout.

%!TEX root = constraint-layout.tex

% \newcommand{\teaserFigure}{
% 	\teaser{
% 		\includegraphics[width=\linewidth]{figures/serengeti-layout.pdf}
% 		\centering
% 	  \caption{\label{fig:teaser}}

% 	}
% }

\newcommand{\serengetiLayoutColumn}{
  \begin{figure}[t!]
    \centering
    \includegraphics[width=0.81\columnwidth]{figures/serengeti-layout-column.pdf}
    {\caption{\label{fig:serengeti-layout}
    The layout for the Serengeti food web using our constraint language, as compared to Baskerville et al.~\cite{baskerville2011spatial}.}}
    \vspace{-40px}
  \end{figure}
}

%%%%%%%%%%%%%%%%%%%%%%%%%%%%%%%%%%
%%%%%%%%%%%%% Design %%%%%%%%%%%%%
%%%%%%%%%%%%%%%%%%%%%%%%%%%%%%%%%%

\newcommand{\smallTreeExample}{
  \begin{figure}[t!]
    \centering
    \includegraphics[width=\columnwidth]{figures/small-tree-example.pdf}
                    {\caption{\label{fig:small-tree-example} The full
                        \projectname\ specification for a small tree with
                        six nodes. Nodes are split into sets based on their
                        depth from the root \texttt{a}, and aligned. A new
                        collection is formed containing the ``layer'' set
                        and the layers are ordered by their depth to form
                        the tree.  }}
    \vspace{-20px}
  \end{figure}
}

\newcommand{\contradictionExample}{
  \begin{figure}[t!]
    \centering
    \includegraphics[width=\columnwidth]{figures/contradiction-example.pdf}
                    {\caption{\label{fig:contradiction-example} (a) The
                        full \projectname\ specification for a small graph
                        with eight nodes. (b) Nodes are aligned based on
                        their \texttt{rank}, and colored based on their
                        \texttt{level}. Two constraints are applied to
                        order the nodes, once by \texttt{level} and once by
                        \texttt{rank}, which produces a contradiction. (c)
                        Nodes are colored based on the amount of error for
                        constraints that are invalid.  }}
    \vspace{-20px}
  \end{figure}
}

%%%%%%%%%%%%%%%%%%%%%%%%%%%%%%%%%%
%%%%%%%%% Demonstration %%%%%%%%%%
%%%%%%%%%%%%%%%%%%%%%%%%%%%%%%%%%%

\newcommand{\krugerLayout}{
  \begin{figure}[t!]
    \centering
    \includegraphics[width=\columnwidth]{figures/kruger-layout.pdf}
    {\caption{
      \label{fig:kruger-layout} A subset of the food web for Kruger 
      National park arranged by trophic level (i.e., carnivore, 
      herbivore, and plant), as seen on the website \cite{kruger2017} 
      and (b) recreated using \projectname.
    }}
    \vspace{-20px}
  \end{figure}
}

\newcommand{\serengetiLayout}{
  \begin{figure*}[t]
    \centering
    \includegraphics[width=\textwidth]{figures/serengeti-layout.pdf}
    \vspace{-20px} {\caption{\label{fig:serengeti-layout} The layout for
        the Serengeti food web using our constraint language, as compared
        to Baskerville et al.~\cite{baskerville2011spatial}. \todo{retake
          photos on retina screen} \todo{label the two sides of the figure
          a/b}}}
  \end{figure*}
}

\newcommand{\serengetiSpec}{
  \begin{figure}[t]
    \centering
    \includegraphics[width=\columnwidth]{figures/serengeti-spec.pdf}
    \vspace{-20px} {\caption{\label{fig:serengeti-spec} The
        \projectname~specification for the Serengeti food web shown in
        Figure~\ref{fig:serengeti-layout}. The code is annotated with the
        number of sets produced, the number of edges added, and the number
        of WebCoLa constraints generated for the final layout.}}
  \end{figure}
}

\newcommand{\syphilisLayout}{
  \begin{figure*}[t]
    \centering
    \includegraphics[width=\textwidth]{figures/syphilis-layout.pdf}
    \vspace{-20px} {\caption{\label{fig:syphilis-layout} The layout for the
        syphilis social network from (a) Rothenberg et
        al.~\cite{rothenberg1998using} as compared to (b) the
        \projectname~layout. \todo{add some padding on the layout of the
          aligned men}}}
  \end{figure*}
}

\newcommand{\syphilisSpec}{
  \begin{figure}[t]
    \centering
    \includegraphics[width=\columnwidth]{figures/syphilis-spec.pdf}
    \vspace{-20px} {\caption{\label{fig:syphilis-spec} The
        \projectname~specification for the syphilis social network shown in
        Figure~\ref{fig:syphilis-layout}. The code is annotated with the
        number of sets produced, the number of edges added, and the number
        of WebCoLa constraints generated for the final layout.}}
  \end{figure}
}

%!TEX root = constraint-layout.tex
\section{Introduction}
With an appropriate graph layout, node-link diagrams can
effectively convey properties of an underlying
network structure, such as hierarchy or connectedness. Such visualizations
are common across a variety of domains, including ecological networks
\cite{hinke2004visualizing,harper2006dynamic,lavigne1996cod,baskerville2011spatial,yodzis1998local,cohen2003ecological,kearney2016blog,benson2016higher,kruger2017},
biological systems
\cite{barsky2008cerebral,shannon2003cytoscape,gehlenborg2010visualization,saraiya2005visualizing,becker2001graph,kojima2007efficient,li2005grid},
and social networks
\cite{scott1988social,rothenberg1998using,fitzpatrick2001preventable,mcelroy2003network,fu2011hiv}, 
among many others. A graph layout may utilize
domain-specific properties in order to emphasize relevant patterns in the
data. 
%In an ecological network for example, nodes could be split by trophic
%level (e.g., its positition in the food chain) to highlight the hierarchy
%of producer/consumer relationships
%(Figure~\ref{fig:kruger-layout}). This visualization shows a customized
%layout: a unique layout that leverages knowledge of the data domain for
%node placement and understanding of the graph structure.
In a biological pathway, nodes can be layered by their
subcellular location to provide an overview of the cellular structure
(Figure~\ref{fig:tlr4-layout}). The additional ``downstream genes'' layer
can further show the outcomes of this network, clustered by molecular function.

\tlrfourLayout

Many domain-specific layout techniques address particular needs
for customized layouts~\cite{barsky2008cerebral,genc2003constrained,shannon2003cytoscape,kearney2017d3,kearney2017ecopath}. These
techniques leverage common structural properties that are significant to the
domain, such as known data hierarchies including cellular 
structure (Figure~\ref{fig:tlr4-layout}) or trophic level
(Figure~\ref{fig:kruger-layout},~\ref{fig:serengeti-layout}), as a guiding property of the
layout. However, these techniques rarely generalize beyond the domain for
which they were designed, and there are numerous
other domains for which customized layouts would be useful
but for which tools that provide them are unavailable.
Thus, when a layout technique does not exist for the
domain of interest, users must either fit their data
to available techniques or design and implement a new algorithm.
Creating a customized layout algorithm requires both
domain and programming expertise, and so introduces a gap between 
analysis needs and the techniques available to handle those needs.

\todo{AB: claimed ``numerous'' other domains.  OK, or overclaiming?}

Constraints can be used to specify desired properties for a layout by
allowing the designer to control the positions of nodes using
domain-specific information.  However, many existing techniques
for constraint layout require the user to define constraints
on individual nodes or node pairs in the graph. This process can be labor
intensive, requiring thousands of similar constraints and careful reasoning
about which nodes should be constrained. Moreover, instance-level constraints 
(e.g., defined \emph{extensionally} via node indices) prevent
reuse of a layout for various graphs from the same domain.

To enable customized domain-specific layouts with reduced
programming effort, we contribute \projectname: a domain-specific language for
specifying high-level constraints for graph layout. Users partition nodes
into sets based on node or graph properties, and apply layout constraints
to these sets. This approach allows users to specify layout
requirements at a high level, deferring the generation of
instance-level constraints to the underlying runtime system. These constraint
definitions reduce specification effort while enabling highly
customized and reusbale graph layouts.

Our implementation of \projectname generates instance-level constraints for
WebCoLa~\cite{WebCoLa}, a JavaScript library for constraint-based graph
layout. To demonstrate the expressiveness of our language, we
recreate several customized layouts found in the scientific research 
literature. We
show that users can compactly specify complex graph layouts that resemble
those produced by customized layout engines, and can reapply these
specifications across different graphs. Our \projectname specifications
can reduce the number of constraints written by the user by up to a 
factor of 90.

%!TEX root = constraint-layout.tex
\section{Related Work}
There are many areas of related work surrounding graph visualization; we
discuss general layout techniques that motivate this work and expand on the
discussion of domain specific layouts.

\subsection{General Graph Layout}
Graph visualization is a long-standing area of research. There are
several common layout techniques for network data, including tree
layouts, Sugiyama-style layouts, and force-directed layouts
\cite{herman2000graph,eades2010graph}. Graphviz~\cite{ellson2001graphviz}
and Gephi~\cite{bastian2009gephi} are two examples of visualization tools
focused on graph layout. D3.js~\cite{bostock:d3} is a JavaScript library
that provides a number of layouts for graph data including
force-directed and hierarchical layouts. WebCoLa~\cite{WebCoLa} is a
JavaScript library for constraint-based layout that can be used alongside
D3 or Cytoscape~\cite{shannon2003cytoscape}. These techniques \jheer{which techniques?} emphasize
certain structural properties \jheer{such as?}, but are agnostic as to the domain \jheer{What do you mean by ``domain''?}. While
WebCoLa can support some customized constraints in addition to more general
structure constraints, the specification of individual inter-node
constraints can be labor-intensive and require additional programming
expertise. Our language aims to reduce the burden of specifying constraints
to produce generalizable layouts.

\jheer{The prose here feels more *tool* focused (e.g., Graphviz, D3, Cytoscape) then *technique* focused (e.g., different layout approaches). I think you should take a step back and think about the big picture here. What are the common automatic (non-constrained based) approaches to node-link diagram layout? What structural features does each show? What are their limitations? Then go on to discuss contrained and/or customized layouts and their value for domain-specific graph layout. The DiG-CoLa paper does a nice job of this, contrasting force-directed and Sugiyama approaches, and showing how DiG-CoLa in some ways enables hybrids of the two. Also, right now the critical discussion of constrained layouts -- their main concepts, strengths, and weaknesses -- is missing.}

In addition to these standard layout techniques, there is a large field of
related work surrounding graph layout techniques that emphasize particular
aesthetic or structural properties in the graph. \textsc{hola}
\cite{kieffer2016hola} produces orthogonal layouts similar
to those produced by hand, informed by common aesthetic properties used in manual layouts. Kieffer et al.~\cite{kieffer2013incremental} present
a constraint-based layout for creating graphs with node and edge alignment,
and demonstrate the feasibility of such a technique in an interactive
system, Dunnart \cite{dwyer2008dunnart}.
\jheer{Weird that a 2013 paper has its feasiblity demonstrated by a 2008 paper...}
These customized layouts reflect
general properties of the graph rather than domain specific details from
the data itself. \jheer{Previous sentence a bit confusing. What domain-specific ``details'' or ``properties''? Perhaps a better description would help.} Our work focuses on providing a general language for
supporting customized graph layouts based on domain specific data
properties with minimal programming expertise.

\todo{Add a specific paragraph/section talking about constraint based techniques}
% Include other resources like IpSepCoLa, DigCoLa, etc. in this new section

\subsection{Domain Specific Graph Visualization}
\todo{Shift from tool to technique for some of these, cerebral is the most
  tool like of any of them} Several tools have been developed to utilize
domain specific information for graph layout, but these tools do not cover
all possible domains of interest. In a blog post describing her approach to
visualizing food webs~\cite{kearney2016blog}, Kearney notes that the node
placement algorithm should \emph{``allow contraining y-position to match
  trophic level while allowing free movement in the x-direction. With no
  such algorithm seemingly readily available, I decided to create my
  own.''} Kearney has developed plugins for D3~\cite{kearney2017d3} and
Ecopath~\cite{kearney2017ecopath} to aid in the visualization of
food webs. Baskerville et al.\ visualized the Serengeti food web
\cite{baskerville2011spatial}; in addition to the static visualization
presented in the paper, they developed an interactive graph
\cite{baskerville2011interactive} created with D3 that approximates this
static version.  Motivated by this related work and the
challenges Kearney describes, our constraint language aims to provide users
with a lightweight strategy for identifying domain specific constraints.

Cerebral \cite{barsky2008cerebral} is a tool designed to visualize
biological systems and support interactive exploration of different
experimental conditions. Cytoscape \cite{shannon2003cytoscape} is a
visualization system designed to explore biomolecular interaction networks
and provides a framework for accepting customized plugins to extend the
system, such as WebCoLa~\cite{WebCoLa}. Each of these tools was designed to
address the needs of domain experts who were unable to find the necessary
support within existing tools. Our work aims to reduce the barrier to
creating customized systems by providing a compact way to specify domain
specific graph layouts that leverage expert knowledge with minimal
programming expertise.

\orange{We will need to extend this section once we've worked out the third example}

% alan's casssowary
% other strategies for writing constriants (constraint based section)

% michael mcguffin - interactive graph modification - genealogical graphs
% david auber - tooklkit tulip 
% frank van ham - perceptual organization of graphs - layout graphs by hand (512)
	% automated don't match humans
% daniel archambeault - graph drawing

% stephen north - graph vis
% bell labs folks - energy minimization, stress majorization

% FOCUS ON the dsl angle - mismatch between user can specify and the
% implementation as constraint program

%!TEX root = constraint-layout.tex
\section{\projectname Constraints and WebCoLa Implementation}
\label{sec:constraints}
We implemented seven constraints for \projectname: 
\texttt{alignment}, \texttt{position}, \texttt{order}, \texttt{circle},
\texttt{cluster}, \texttt{hull}, and \texttt{padding}. In this work, we
target  Dwyer et al.'s WebCoLa library \cite{WebCoLa} in order to support
interactive, web-based layouts. In this section, we describe the requirements 
for each type of constraint and how we implemented them using WebCoLa. 
In WebCoLa, constraints are defined based on the \texttt{\_id} of the
node in the graph specification. We leverage two of WebCoLa's
constraint types for our implementation: \emph{alignment} constraints 
and \emph{position} constraints. For other \projectname constraints, we
approximate the behavior using additional edges or by applying padding
to the nodes in the layout.

%%%%%%%%%%%%%%%%%%%%%%%%%%%%%%%%%%%%%%%%%%%%%%%%%%%%%%%%%%%%%%
\subsection{Alignment Constraints}
\constraint{\texttt{align} \emph{x} \texttt{axis}}
Alignment constraints ensure that all nodes in the set share one of their
coordinates. For this implementation, the constraint must specify the \texttt{axis}
as either \texttt{x} or \texttt{y}. These constraints are defined as:

\begin{definition}
Let $S$ be the set on which the constraint \texttt{align x axis} is applied.
For all nodes $n_1$ and $n_2$ in $S$ such that $n_1 != n_2$, we define 
a constraint such that $n_1[y]$ == $n_2[y]$.
\end{definition}

\todo{Is it helpful to have a description like this? Is there a better way
to write this up?}

The user may also optionally identify an \texttt{orientation} for the alignment.
The orientation is useful for producing alignments when the size of the 
elements do not match. By default, the orientation is defined as \texttt{center} 
with the behavior described above. When the axis is defined as \texttt{x} 
the user may specify either \texttt{top} or \texttt{bottom} which introduces 
an offset to align the top or bottom of the nodes. For \texttt{y} axis
alignment, the user may specify either \texttt{right} or \texttt{left}.

\emph{Implementation.}
Alignment constraints are one of the constraint types natively supported in
WebCoLa. The WebCoLa alignment constraint takes the \texttt{\_id} of all nodes
that should be aligned and offsets for each node that enable changes to the
alignment orientation of the nodes.

\emph{Rationale.} Support for alignment on either the x or y axis allows
users to produce layouts that exhibit grid like layout properties similar
to those produced by Kieffer et al.\ \cite{kieffer2016hola}.

%\jheer{Why only x/y axes? Why not polar coordinates, for example? Or along arbitrary line segments defined as guides? I'm not saying you *should* support these, only that it's not clear why you focus where you do. Also I don't totally understand what alignment means here (nor what the default would be if unspecified). Readers may be confused about what you mean by applied to nodes vs. applied to sets, as earlier we talk about sets as the units that constraints are applied to. Also, centroid or boundary, which is used when and why?}

% When applied to sets, this constraint produces an alignment based on the centroid or boundary of the set elements.
% \todo{We don't really have alignment working on set elements, only on node elements.}

%%%%%%%%%%%%%%%%%%%%%%%%%%%%%%%%%%%%%%%%%%%%%%%%%%%%%%%%%%%%%%
\subsection{Position Constraints}
\constraint{\texttt{position} \emph{right} \texttt{of} \emph{``top\_guide''}}
Position constraints ensure that all nodes in the set are positioned relative to
a guide \orange{or previously named set}. The user most specify the relative
orientation for the position as one of \texttt{left}, \texttt{right}, 
\texttt{above}, or \texttt{below}. These constraints are defined as:

\begin{definition}
Let $S$ be the set on which the constraint \texttt{position left of guide g} is applied.
For all nodes $n$ in $S$, we define a constraint such that $n[x]$ < $g[x]$.
\end{definition}

The user may optionally prove a \texttt{gap} that enforces additional spacing
between the nodes and identified guide. This property produces the following
definition:

\begin{definition}
Let $S$ be the set on which the constraint \texttt{position left of guide g with gap 20} is applied.
For all nodes $n$ in $S$, we define a constraint such that $(n[x]$ + 20) < $g[x]$.
\end{definition}

\emph{Implementation.}
Position constraints are one of the constraint types natively supported in
WebCoLa and also include a \texttt{gap} for the constraint. For each node
in set $S_1$, we produce one position constraint relative to the specified guide.
\orange{When the position constraint is defined relative to a named set $S_2$,
we produce one position constraint for each pair of nodes between $S_1$ and $S_2$.}

\emph{Rationale.} Position constraints allow the user to provide
overarching constraints relative to global elements such as guides. These
can be used to constrain the overall size of the graph or to section off
different areas.

%\jheer{Why only relative guides? Why can't relative position constraints be applied between sets?}
% \todo{This isn't implemented, but probably could be fairly easily. Orange above.}

%\todo{position constraints on set elements}

%%%%%%%%%%%%%%%%%%%%%%%%%%%%%%%%%%%%%%%%%%%%%%%%%%%%%%%%%%%%%%
\subsection{Order Constraints}
\constraint{\texttt{order} \emph{y} \texttt{axis} \texttt{by} \emph{``depth''}}
Order constraints enforce a sort order on the set elements. The user must 
specify the \texttt{axis} as either \texttt{x} or \texttt{y} and must define
the node property \texttt{by} which the order is determined. These 
constraints are defined as follows:

\begin{definition}
Let $S$ be the set on which the constraint \texttt{order x axis by node depth} is applied.
For all nodes $n_1$ and $n_2$ in $S$, we define a constraint such that $n_1[x]$ < $n_2[x]$
if $n_1[depth] < n_2[depth]$.
\end{definition}

The user can optionally define an explicit list of values for a custom
\texttt{order}; otherwise, the elements are ordered alphabetically \texttt{by}
the specified property. The user may also optionally include a boolean of whether
or not to \texttt{reverse} the order. Users may finally specify an optional
\texttt{band} property for the constraint that determines a size for each
set region. For the \texttt{band} property, users can introduce fixed spacing
to the order of the elements by introducing boundaries between elements.
Constraints with the \texttt{band} property are defined as follows:

\begin{definition}
Let $S$ be the set with $N$ elements on which the constraint 
\texttt{order x axis by node depth with band 200} is applied.
We define $N+1$ boundary guides: $b_1, b_2, ..., b_{N+1}$ where
$b_1[x] + 200 == b_2[x]$. We then sort the elements by the depth.
Let $n_1$ be the element with the smallest depth and $n_2$ be the element 
with the next smallest depth. We then produce constraints such that
$b_1[x] < n_1[x]$ and $n_1[x] < b_2[x]$ and $b_2[x] < n_2[x]$.
\end{definition}

\emph{Implementation}
We define order constraints by defining WebCoLa position constraints based
on the definitions defined above. \todo{Confirm definitions/implementation.}

%\jheer{Similar question as above regarding x/y vs. other possible options.}

\emph{Rationale.} Unlike position constraints, order constraints allow the
user to specify the overall sort order for the layout.

%%%%%%%%%%%%%%%%%%%%%%%%%%%%%%%%%%%%%%%%%%%%%%%%%%%%%%%%%%%%%%
\subsection{Circle Constraints}
\constraint{\texttt{circle} \texttt{around} \emph{``center''}}
Circle constraints allow the user to specify a ring layout for a set of
elements. The constraint must include which element the set elements should
be positioned around or may specify ``center'' for a generic center point
to be used.

\todo{Formal definition of this constraint?}

\jheer{I'm confused regarding the center. What is a generic center point? Can one specify a x,y point or related guide? What does it mean to put an element in the center? Does this constraint work with sets of sets?}

\emph{Implementation.}
Circle constraints are not a supported constraint type in WebCoLa. To 
demonstrate the utility of this constraint type we approximated the behavior
in our WebCoLa implementation. To do this, we first add a temporary node
to act as the center of the circle layout. We then add a link between each 
node in the set and the temporary center. \todo{Finish and confirm description of procedure.}

\todo{I've looked into two ways of approximating the circle procedure. 
(1) Add edges between each node in the circle and the center, and around
the circle edge. (2) Produce the layout with position constraints and 
then convert to polar coordinates.}

\emph{Rationale.} Circular structures are a common layout requirement of
existing strategies \todo{citation} and can be crucial for relating
properties of the underlying structure.

%%%%%%%%%%%%%%%%%%%%%%%%%%%%%%%%%%%%%%%%%%%%%%%%%%%%%%%%%%%%%%
\subsection{Cluster Constraints}
\constraint{\texttt{cluster}} 
Cluster constraints encourage a organic clustering of the nodes around
a center point.

\todo{Formal definition of this constraint?}

\emph{Implementation.}
Cluster constraints are not currently supported in WebCoLa. In order to
produce a clustered appearance, we add temporary edges between all nodes
in the set to produce a clique and require the edges to have shorter length.

\emph{Rationale.}
Cluster constraints do not have as strong an impact as alignment or
position constraints, but encourage the nodes to be attracted to one
another to produce a more inherent grouping than may otherwise occur.
This behavior is similar to many force-directed techniques in which
nodes are attracted to others of similar type.

%%%%%%%%%%%%%%%%%%%%%%%%%%%%%%%%%%%%%%%%%%%%%%%%%%%%%%%%%%%%%%
\subsection{Hull Constraints}
\constraint{\texttt{hull}} 
Hull constraints create a boundary around the set elements and prevents
elements from other sets from passing through this boundary. These 
constraints are defined as follows:

\begin{definition}
Let $S$ be the set on which the constraint \texttt{hull} is applied. We produce a
rectangle $B$ with properties $B.x1$, $B.x2$ $B.y1$, $B.y2$. For all nodes
$n$ in $S$, we define constraints such that $B.x1 < n[x]$, $n[x] < B.x2$,
$B.y1 < n[y]$, $n[y] < B.y2$. For all nodes $m$ \emph{not} in $S$, we define
constraints such that $m[x] < B[x1] || B[x2] < m[x]$ and $m[y] < B[y1] || B[y2] < m[x]$.
\end{definition}

\emph{Implementation.}
We generate hull constraints in WebCoLa using its built-in support for 
specifying \texttt{groups} in the layout which produce a boundary around 
the nodes defined by their \texttt{\_id}.

\emph{Rationale.}
Hull constraints allow users to create enclosed regions of the graph in
which the set elements do not interact with elements outside the set. This
type of constraint can prevent nodes from interleaving when they are part
of separate sets.


%%%%%%%%%%%%%%%%%%%%%%%%%%%%%%%%%%%%%%%%%%%%%%%%%%%%%%%%%%%%%%
\subsection{Padding Constraints}
\constraint{\texttt{padding 5}} 
Padding constraints introduce an amount of space around the element relative
to other elements in the set. These constraints are defined as follows:

\todo{Formal definition of this constraint?}

\emph{Implementation.}
Padding introduces additional space around nodes without constraining the
axis on which the padding is added. Our current implementation adds a
padding to the nodes which essentially increases the size of the element
when WebCoLa's non-overlap behavior is applied. However, in this implementation
padding can only be specified once and applies to all nodes, not just nodes
in the specified set. Future work is required to develop constraint that
respect the padding only relative to set elements.

\emph{Rationale.}
To support the overall legibility of the graph, additional padding can be
beneficial to ensure that nodes are not placed to close together.

\subsection{Creating Guides in WebCoLa}
\constraint{$O(g)$ new nodes}
In \projectname, the user may define guides to control the graph layout.
To use these guides in WebCoLa constraints, we add a new node to the graph
for each guide and generate constraints relative to this node. These 
temporary nodes are included in WebCoLa's layout but are hidden from the
final visualization of the graph layout.

\subsection{Application of Multiple Constraints}
\todo{Describe the interplay of different constraints for the specification.}

%!TEX root = constraint-layout.tex
\section{Implementation of \projectname~using WebCoLa}
We created an implementation of \projectname\ in which we generate node
specific constraints in WebCoLa~\cite{WebCoLa} based on our high-level
constraints. WebCoLa is an open-source JavaScript library for
creating high-quality, stable constraint layouts, and was therefore a
useful back end to demonstrate the utility of \projectname\ on a number of
examples (see Section \ref{sec:examples}). In this section, we discuss the
constraint generation process for producing WebCoLa constraints and the 
addition of built-in properties of the graph structure.

%%%%%%%%%%%%%%%%%%%%%%%%%%%%%%%%%%%%%%%%%%%%%%%%%%%%%%%%%%%%%%
\subsection{Generating WebCoLa Constraints}

\todo{Describe the constraint generation procedure for within and between
  sets as well as the number of constraints that each one generates.}

%%%%%%%%%%%%%%%%%%%%%%%%%%%%%%%%%%%%%%%%%%%%%%%%%%%%%%%%%%%%%%
\subsection{Built-In Properties of the Graph Structure}
In addition to defining constraints relative to properties arising from
the domain, it may also be important to define constraints on properties
of the graph structure. In our WebCoLa implementation, we support a 
number of built-in properties for the nodes that reflect the graph structure.
These properties are automatically computed and added to the graph 
specification when they are used in one of the \projectname\ constraints. 
These properties are only computed if such a property does not
already exist on the nodes and are subject to a number of expectations
regarding the graph input; for graphs that do not meet these expectations,
users are shown a warning and required to compute the properties
themselves.

\begin{description}
\item[\texttt{\_id}] The node index in the graph specification. This
  property is always computed regardless of whether or not it is used.
\item[\texttt{depth}] One more than the max depth of the node's
  parents. This property is only computed for graphs that contain only one
  root node and do not contain cycles. \todo{Is there a definition for
  graphs with multiple roots (or can we handle disconnected graphs)? Can
  the user specify the root in the specification?}
\item[\texttt{parent}] The parent of the current node. This property is
  only allowed if the node has one parent; otherwise, it is defined as the
  parent with the smallest \texttt{\_id}.
\item[\texttt{firstchild}] The child node of the current node with the smallest \texttt{\_id}.
\item[\texttt{incoming}] The list of nodes that have edges where the
  current node is the target (e.g., all parent nodes).
\item[\texttt{outgoing}] The list of nodes that have edges where the
  current node is the source (e.g., all child nodes).
\item[\texttt{neighbors}] The list of nodes that have edges connected to
  the current node. This property is the join of the \texttt{incoming} and
  \texttt{outgoing} properties.
\item[\texttt{degree}] The number of \texttt{neighbors}.
\end{description}

\todo{confirm all descriptions}

We selected these properties as common structural elements that could be
applicable to layout specifications. For example, the \texttt{depth}
property is useful for producing hierarchical tree layouts and the 
\texttt{incoming}, \texttt{outgoing}, and \texttt{neighbors} properties
reflect structure produced by the graph edges. There are many other
properties that could be useful for graph layouts that are not included 
here; this list could easily be extended in the future to include other
common properties. Furthermore, additional properties can always be
computed by the user and added to the graph prior to running the layout.
%!TEX root = human-constraint-layout.tex
\section{Results}
Now that we have described the design and implementation of our high-level constraint language, we will demonstrate the utility of this technique through a number of example use cases. These examples were chosen to highlight both common layout techniques and highly domain specific layouts. We show the results of our layouts \blue{and provide runtime information for each phase of the process.}

% \setlength{\tabcolsep}{0.5em} % for the horizontal padding
% {\renewcommand{\arraystretch}{1.6} % for the vertical padding
% \begin{table*}[]
% \centering
% \begin{tabular}{r|c|c|c|c}
%                      & Figure                                      & Nodes & HvZ & WebCoLa     \\ \hline
% Small Tree           & \ref{fig:small-tree}, \ref{fig:tree-layout} & 6     & 2   & 14      \\ \hline
% Medium Tree          & \ref{fig:tree-layout}                       & 47    & 2   & 801      \\ \hline
% Large Tree           & \ref{fig:tree-layout}                       & 80    & 2   & 2736

% \end{tabular}
% \caption{\orange{Fill this in!}}
% \label{tab:numConstraints}
% \vspace{-20px}
% \end{table*}

\subsection{Tree Layouts}
\treeLayout
\tlrFourLayout
Trees are one example of a common layout technique that is often addressed in visualization tools \blue{cite D3, graphviz, etc.}. The basic tree layout from Figure \ref{fig:small-tree} is shown on three trees of varying sizes in Figure \ref{fig:tree-layout}.

While our constraint language can produce highly customized layouts, the layouts are also reusable across specifications allowing users to easily reapply the same layout to multiple graphs.

\subsection{Food Webs}
\smallFoodWebLayout
Food webs visualize complex producer-consumer relationships in ecological systems and are a common presentation strategy for this information despite the challenge of creating an informative visualization \blue{citations}. There are many examples of small food webs, which could easily be created by hand, but we show a simple constraint specification in Figure \ref{fig:small-foodweb-layout}. However, this is just a small example of a food web, whereas the webs are often much more complex. \green{Include a larger example of a food web layout that we can compare to.}

\subsection{Biological Networks}
Cerebral \cite{barsky2007cerebral} is a system for the layout and analysis of biological systems. The authors note that existing ``tools did not fully meet the needs of our immunologist collaborators'' as motivation for the design of their system. In order to examine the feasibility of our constraint language for highly domain specific tasks, we recreated the TLR4 graph shown in \cite{barsky2007cerebral} using our language. The constraint specification, layout, and Cerebral graph are shown in Figure \ref{fig:tlr4-layout}.

This constraint specification demonstrates a number of the techniques supported by our constraint language. In particular, the user adds an \texttt{"exclude"} statement to remove the nodes with an \texttt{"unknown"} type from the layout (these are the gray nodes in teh graph). To produce the hierarhical layout that reflects the underlying biological system, the user simply applies a custom ordering on the types of the nodes for this graph. The main difference between our layout and the one prodcued by Cerebral is that our layout more strongly penalizes vertical distance which causes the graph to compress into thinner layers. Future work will explore how to better tune these properties to provide unconstrainted movement of nodes within the bounds provided by our constraints.

%!TEX root = constraint-layout.tex
\section{Limitations \& Future Work}

% constraints not satisfied and debugging and not have constraints respected by underlying solver

\todo{3. Talk about the need to pre-clean and post-edit the graphs. In particular, discuss the fact that this process is assuming that the graphs contain all the information we need as attributes. As such, we can defer some of the complexity that might come out of more complex selection techniques. Also discuss that we added some additional labeling to them in order to create the final 'figure'. But we did \emph{not} edit the layout.}

\todo{4. Talk about decision to allow nodes to occur in multiple sets in the creation process (which may result in the user specifying constraints that lead to contradictions.)}

% wrangler/polestar/lyra strategies for specifying constraints
% dsl leveraged as useful interface for processing logic

\subsection{Limitations of WebCoLa as the Constraint Solver}

Our implementation with WebCoLa allows us to demonstrate the utility of \projectname~for creating customized, generalizable layouts. However, our implementation was constrained by what WebCoLa currently supports. For example, we were unable to express \projectname's circle constraints in WebCoLa as it does not currently include such a constraint type, though it has been identified in the wiki as an area of future work \orange{citation?}. This work contributes new strategies for the specification of graph layout constraints, but does not aim to create a highly optimized constraint solver for the language; Future work should explore the design of a constraint solver optimized for \projectname~and should look at how new or existing algorithms might co-evolve alongside the high-level language.

\todo{Talk about fact that circle constraint is not particularly well defined in our implementation, but approximates the layout in webcola as an example. or just exclude it from the implementation as something we cannot express?}

We also encountered some behavioral mismatches between the implementation of WebCoLa and our expectations for the graph layout. For example, WebCoLa utilizes a default link length for the layout which attempts to optimize node positions to be as close to the default link length as possible. This technique is a common strategy for force-directed layouts \orange{citation} and helps to highlight the underlying structure of the graph. However, for many of our layouts that require larger layout changes independent of this graph structure, the edge length optimization could produce layouts that were undesirable (\orange{Figure ?}).

\todo{Talk about non-overlap problem for node layout (e.g., it is applied at the start of the layout and it can make it hard for nodes to follow the layout as they cannot pass over each other)}

%!TEX root = constraint-layout.tex
\section{Conclusion}
We present \projectname: a domain-specific language for specifying high-level
constraints for customized graph layouts. \projectname enables concise 
specification of graphs by allowing users to apply constraints over sets
of nodes rather than on individual nodes. These customized layouts can
be easily reapplied to different graphs that share domain-specific properties.
We implemented \projectname in WebCoLa and demonstrate the expressiveness
of this technique for several real-world examples from ecological networks,
biological systems, and social networks. These specifications reduce the
number of constraints written by the user while still enabling flexible
constraints for the graph data.

\section*{Acknowledgements}
%\todo{Write acknowledgements section}
\textsc{Removed from anonymized submission.}

\todo{Remember to anonymize whole paper (this includes link to source if we include one).}
\todo{Update the title?}



%% Acknowledgments
\begin{acks}                            %% acks environment is optional
                                        %% contents suppressed with 'anonymous'
  %% Commands \grantsponsor{<sponsorID>}{<name>}{<url>} and
  %% \grantnum[<url>]{<sponsorID>}{<number>} should be used to
  %% acknowledge financial support and will be used by metadata
  %% extraction tools.
  % This material is based upon work supported by the
  % \grantsponsor{GS100000001}{National Science
  %   Foundation}{http://dx.doi.org/10.13039/100000001} under Grant
  % No.~\grantnum{GS100000001}{nnnnnnn} and Grant
  % No.~\grantnum{GS100000001}{mmmmmmm}.  Any opinions, findings, and
  % conclusions or recommendations expressed in this material are those
  % of the author and do not necessarily reflect the views of the
  % National Science Foundation.
  The author would like to thank Alan Borning, Jeffrey Heer, and the students of UW CSE599F1 for their feedback and support on this work.
\end{acks}


%% Bibliography
\bibliography{references}


%% Appendix
%\appendix
%\section{Appendix}

%Text of appendix \ldots

\end{document}
