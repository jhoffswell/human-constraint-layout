%!TEX root = human-constraint-layout.tex
\section{Introduction}
\alignedTree
Graphs are common across a number of domains including social network analysis \cite{scott1988social,travers1967small,granovetter1973strength,watts1998collective,freeman1978centrality}, biological systems \cite{barsky2008cerebral,shannon2003cytoscape,gehlenborg2010visualization,saraiya2005visualizing}, and ecological networks \cite{hinke2004visualizing,harper2006dynamic,lavigne1996cod,baskerville2011spatial,yodzis1998local,cohen2003ecological,kearney2016blog,benson2016higher}. In order to highlight relevant trends in the data, it is important for the graph layout to utilize domain specific details. In response, various domain specific layout tools have been developed \cite{barsky2008cerebral,shannon2003cytoscape,kearney2017d3,kearney2017ecopath,gawron2016pss}.

When a graph layout tool does not exist for the domain of interest, the user is required to either fit their data to one of the available tools or create a customized layout. Creating such a customized layout often requires both domain and programming expertise. When a collaboration is infeasible, the domain expert may be required to invest large amounts of time into learning the skills necessary to program the customized layout. Furthermore, these tools are hard to generalize outside their domain as they are carefully crafted by design to meet the needs of that particular domain, which means that it can be hard for domain experts to share their new programming expertise to others in their field. This barrier to creating highly customized, domain specific layouts means there is a gap between the analysis needs of some domains and the availability of tools to handle those needs.

In response, we present a high-level constraint language for graph layout. This approach allows the user to easily create a new layout customized for her needs, while deferring calculation of the layout to the underlying system. The user focuses on expressing the main layout properties of interest by specifying constraints over sets of node in the graph. Our system then converts these constraints to inter-node constraints for WebCola \cite{WebCoLa}, a Javascript library for constraint based graph layout. The user can then tune properties of the layout by modifying preferences for WebCola or otherwise modifying the graph with D3 \cite{bostock:d3}. We show that users can compactly specify complex graph layouts that resemble those produced by customized layout engines. Finally, we discuss design considerations and areas for future work.



%  