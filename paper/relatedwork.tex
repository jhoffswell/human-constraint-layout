%!TEX root = human-constraint-layout.tex
\section{Related Work}
There are many areas of related work surrounding the visualization and layout of graphs, and domain specific graphs in particular. Graphs are a common type of data seen across a variety of domains including social networks \cite{scott1988social,travers1967small,granovetter1973strength,watts1998collective,freeman1978centrality}, biological systems \cite{barsky2008cerebral,shannon2003cytoscape,gehlenborg2010visualization,saraiya2005visualizing}, and ecological networks \cite{hinke2004visualizing,harper2006dynamic,lavigne1996cod,baskerville2011spatial,yodzis1998local,cohen2003ecological,kearney2016blog,benson2016higher}. In this section, we identify some common tools for graph layout and describe some tools that have been tailored for domain specific tasks.

\subsection{Graph Visualization}
Graph layout has been a long-standing area of research and there are a number of common layout techniques for visualizing graph data including tree layouts or force-directed layouts. Graphviz \cite{ellson2001graphviz} and Gephi \cite{bastian2009gephi} are two examples of visualization engines specific to graph layout. D3.js \cite{bostock:d3} is a JavaScript library that provides a number of built in layouts for graph data including force-directed and hierarchical layouts. WebCoLa \cite{WebCoLa} is a JavaScript library for constraint-based layout that can be used alongside D3 or Cytoscape~\cite{shannon2003cytoscape}.

In addition to these standard layout techniques, there is a large field of related work surrounding graph layout techniques that emphasize particular aesthetic or structural properties in the graph. HOLA \cite{kieffer2016hola} presents a layout engine to produce layouts similar to those produced by hand by identifying common aesthetic properties of the graph used in manual layouts and using the results to drive the design of a new layout algorithm. Kieffer et al. \cite{kieffer2013incremental} present a constraint-based layout for creating graphs with node and edge alignment, and demonstrate the feasibility of such a technique in an interactive system, Dunnart \cite{dwyer2008dunnart}.

While these are some examples of customized layouts, the customization reflects general properties of the graph as opposed to domain specific details. Our work focuses on providing a general language for supporting customized graph layouts with minimal programming expertise.

\subsection{Domain Specific Graph Visualization}
To address the domain specific needs for graph analysis, a number of tools have been developed to utilize domain specific information in the layout. Cerebral \cite{barsky2008cerebral} is a tool designed to visualize biological systems and support interactive exploration of different experimental conditions. Cytoscape \cite{shannon2003cytoscape} is a visualization system designed to explore biomolecular interaction networks and provides a framework for accepting customized plugins to extend the system. Becker and Rojas \cite{becker2001graph} provide a customized algorithm for visualizing metabolic pathways. Kearney has developed D3 \cite{kearney2017d3} and Ecopath \cite{kearney2017ecopath} plugins to aid in the visualization of food webs.

Each of these tools was designed with a particular domain in mind to address the needs of domain experts who were unable to find the necessary support within existing tools. Our work aims to reduce the barrier of creating these customized systems by providing a compact way to specify domain specific graph layouts.
