%!TEX root = human-constraint-layout.tex
\newcommand{\smallTree}{
  \begin{figure}[t]
    \centering
    \includegraphics[width=\columnwidth]{figures/small-tree.pdf}
    \vspace{-20px}
    {\caption{\label{fig:small-tree} A simple constraint specification for a tree layout and the resulting layout on a six node tree.}}
  \end{figure}
}

\newcommand{\configurationPanel}{
  \begin{figure}[t]
    \centering
    \includegraphics[width=\columnwidth]{figures/configuration.pdf}
    \vspace{-20px}
    {\caption{\label{fig:config-panel} In the configuration panel, users can modify properties of the development environment and graph rendering.}}
  \end{figure}
}

\newcommand{\debuggingPanel}{
  \begin{figure}[t]
    \centering
    \includegraphics[width=\columnwidth]{figures/debugging.pdf}
    \vspace{-20px}
    {\caption{\label{fig:debug-panel} In the debugging panel, users can view the sets and WebCoLa constraints generated from their high-level constraint specification.}}
  \end{figure}
}

\newcommand{\schemaPanel}{
  \begin{figure}[t]
    \centering
    \includegraphics[width=\columnwidth]{figures/schema.pdf}
    \vspace{-20px}
    {\caption{\label{fig:schema-panel} In the schema pane, users can view the schema for our high-level constraint language.}}
  \end{figure}
}