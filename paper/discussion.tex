%!TEX root = human-constraint-layout.tex
\section{Discussion and Future Work}
In the previous section we showed a few example use cases of our constraint language for various types of graph layout. In this section we talk about some of the lessons learned from this process, our current implementation strategies, and areas for future work on this project.

\subsection{Adding Customized Annotations}
Creating a domain specific layout generally requires both a domain expert and individual with programming expertise. The domain expert provides knowledge about the desired layout motivated by the underlying graph properties whereas the programmer provides the expertise to actually compute the desired layout algorithm. In some cases, the domain expert may also strive to act as the programmer, but taking on such a role may require large amounts of time and effort on behalf of the domain expert.

The goal of this work is to provide a compact language in which to specify domain specific layouts. In the previous section, we showed that we can feasibly recreate complex domain specific layouts with a compact layout specification. However, as is evident from the comparisons (Figure \ref{fig:serengeti-layout}, \ref{fig:tlr4-layout}), the user may want to incorporate additional information via node labeling, group labeling, or other annotations on the visualization. Our system does not currently provide support for such customizations. Future work may want to examine how to incorporate layout driven annotations.

\subsection{Constraint Specification Syntax}
In future iterations of this constraint language, we need to revisit some areas of the current constraint syntax to ensure that constraints are both understandable and expressive. There are two aspects of the current syntax that require additional work: specifying constraints and specifying sets.

\subsubsection{Specifying Constraints}
We identified two styles of high-level constraints, \emph{within set} constraints and \emph{between set} constraints, which represent different techniques for applying constraints over sets of nodes (see Section \ref{sec:sets}). In an earlier iteration of our constraint language, we defined a high-level constraint as follows:

\begin{verbatim}
        {
          "name": NAME,
          "set": SET,
          "within": [CONSTRAINTS...],
          "between": [CONSTRAINTS...]
        }
\end{verbatim}

In our final constraint language, we decided to standardize the specification of layout constraints using the \texttt{"constraint"} syntax rather than \texttt{"within"} and \texttt{"between"} in order to support more customizable constraint specifications, instead varying the type of constraints via the \texttt{"set"} or \texttt{"from"} definition for the high-level constraint. However, in all of the examples shown in this paper, there is a tight relationship between the sets used for \emph{within set} and \emph{between set} constraints (Figure \ref{fig:small-tree}, \ref{fig:tree-layout}, \ref{fig:serengeti-layout}, \ref{fig:small-foodweb-layout}, \ref{fig:tlr4-layout}). Future work is required to understand if standardization is appropriate or if a syntax like the one shown here would be more understandable.

\subsubsection{Specifying Sets}
Future work is also required to explore how best to specify sets in our language. We described a number of techniques for specifying sets including \emph{partitioning} and \emph{manual specification} via predicates (see Section \ref{sec:sets}).

For the \emph{partitioning} approach the user can include additional properties \texttt{"include"} and \texttt{"ignore"} to modify which nodes are selected for a given set and which sets are excluded from the partitioning; however, while these properties sound similar from a linguistic standpoint, it is important to note that they do not operate on the same aspects of the graph and thus do not have reflexive behaviors. Each of these properties was initially introduced to target particular use cases for graph layout, but in future iterations it may be useful to revisit the set specification to create a more straightforward and understandable syntax.

The process of \emph{manual specification} for sets using the \texttt{"expr"} syntax also introduces a potential problem in that the sets do not enforce a disjoint partition of the nodes; in other words, it is currently possible to specify sets using this syntax that include overlapping nodes. This limitation means that it is possible for users to specify constraints that are impossible to satisfy. Additional research is required to determine if there are benefits from supporting overlapping sets or if the manual specification syntax should warn against or disallow such specifications from occurring.

\subsection{Generating Constraints and Optimizations}
The current process of generating WebCoLa constraints from our high-level constraints is unoptimized (see Table \ref{tab:numConstraints}). There are a few optimizations that can be made in particular that can help to reduce the number of WebCoLa constraints.

\emph{Between set alignment} constraints currently force all the nodes across all the sets to become aligned. An improvement would be to use an iterative layout that produces the inter-set layout before positioning the set as a whole. This strategy would significantly reduce the number of constraints that need to be solved by the underlying system and produce a more expected layout for the specified constraint.

\emph{Order} constraints currently produce constraints between pairs of nodes, ignorant of the desired ordering. One optimization is to reflect the ordering in the constraint generation process to only produce constraints for neighboring nodes. For \emph{within set} constraints, this strategy would only apply constraints between adjacent nodes in the final order. For \emph{between set} constraints, we could include an optimization to use temporary nodes to enforce the order rather than combinatorially producing constraints between pairs of nodes.

\emph{Between set position} constraints will produce many redundant constraints because of the current constraint generation in which the layout first partitions pairs of nodes that require a constraint, and then applies the constraint to each node in the pair. \emph{Position} constraints currently produce the same effect for both \emph{within set} and \emph{between set} constraints since the position is relative to a particular node. However, because of the underlying set generation procedure, \emph{between set} constraints are significantly less optimal than \emph{within set} constraints (Table \ref{tab:numConstraints}). Future work is required to examine how to optimize this process and what distinctions should be expected between these two types of the constraint.

\subsection{Debugging and Defining Constraints}
We provide a small amount of debugging support in the current iteration of the system, but we would like to continue exploring improved support for understanding the output of our high-level constraint syntax. When viewing the output of a graph there are a few questions that the user might have: are the constraints generated from my specification what I expected (e.g. correct)? If not, what went wrong and what should I do next? There are a number of possible techniques that can be employed to help users better understand the output of this system and define new constraints.

\subsubsection{Previewing the Graph Layout}
The current process of specifying graphs relies heavily on clustered abstractions of the underlying nodes. To specify constraints, it may be beneficial for users to dynamically change the coloring of nodes in the graph to show different properties and thus help select the partitions of interest. It may also be beneficial to show approximations of the graph layout by creating a preview using the node abstractions. This technique could help the user understand the high-level relationships for \emph{between set} constraints, while deferring understanding of \emph{within set} constraints to after the final layout has been computed.

\subsubsection{Determining Constraint Correctness}
While developing examples for this work, we found that there were some instances where the low-level WebCoLa constraints generated by the system would be correct, but that WebCoLa did not perform sufficient iterations to produce a graph layout that accurately reflects the constraints. In future iterations of this work, we would like to provide a way for users to determine how well the final layout reflects the underlying (and high-level) constraints by computing how many of them are satisfied by the final layout. Allowing users to pin nodes or areas of the graph that are correct, while restarting the layout on other may allow users to iteratively examine and correct the layout to produce a result that matches their expectations.

\subsection{Evaluation}
Finally, for the next version of the system we would like to perform evaluations with domain experts to examine the utility of this system for creating domain specific layouts. We would like to assess how easy it is to learn and use this system for producing complex graph layouts and how expressive the language is for different types of desirable layouts.
