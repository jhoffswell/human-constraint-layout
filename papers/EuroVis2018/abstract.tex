%!TEX root = constraint-layout.tex
\newcommand{\paperabstract}{
	
	Constraints enable flexible graph layouts that combine the ease of automatic layouts with designs customized to the particular domain. However, constraint-based layouts often require individual constraints over node pairs. Such constraints are specific to the particular graph and thus cannot generalize to other graphs in the same domain. To facilitate the specification of customized and generalizable constraint layouts, we contribute \projectname: a domain-specific language for specifying high-level constraints relative to properties of the backing data. Users identify node sets based on data or graph properties and apply high-level constraints within each set of nodes. Applying constraints to node sets rather than on individual nodes reduces specification effort. \projectname~also facilitates reapplication of customized layouts across distinct graphs by separating the constraint specification from individual properties of the nodes. We demonstrate the conciseness, generalizability, and expressiveness of \projectname~on a series of real-world examples from ecological networks, biological systems, and social networks.

  \begin{classification}
  \CCScat{Human-centered computing}{}{Visualization}{Graph drawings}
  \end{classification}
}
