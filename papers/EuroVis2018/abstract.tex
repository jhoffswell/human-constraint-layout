%!TEX root = constraint-layout.tex
\newcommand{\paperabstract}{
	
	\todo{dive right into constrained layout techniques and discuss that our technique combines automatic layout and customized design. move larger funnel to intro}

  Customized graph layouts can highlight unique structural information indicative of the domain. However, such customized layouts rarely exist for every possible domain of interest, thus requiring collaboration, compromise, or computing education. To support the design of highly customized, domain specific layouts, we contribute a high-level constraint language for graph layout. Users identify node sets based on data or graph properties, and specify constraints that apply within or between these sets. Applying constraints to node sets minimizes specification effort on the part of the user and facilitates reapplication of customized layouts across distinct graphs. We demonstrate the simplicity and generalizability of these customized graph layouts on a series of real-world examples from ecological networks, biological systems, and \orange{----?----}.

  \begin{classification}
  \CCScat{Human-centered computing}{}{Visualization}{Graph drawings}
  \end{classification}
}

% More formal definition of 'custmoized graph layout' - often achieved with manual customization
% Definitely need to have crisp definition
% More targeted at end-user perception

% Maybe don't play up the education / expertise angle
% Mismatch between low level constraints 

% computing effort on part of user -> specification effort, computer is the computing element



% Start from constrained layout techniques - combine automatic layout and customized design
% Require individual node/edge constraints; scalability problem
% mismatch in graph data domain and implemntation
%%% Abstract, more pinpointed, but get be more abstract in the intro

