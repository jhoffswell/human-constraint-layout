%!TEX root = constraint-layout.tex
\newcommand{\paperabstract}{
	
	Constraints enable flexible graph layouts that combine the ease of automatic layouts with designs customized to the particular graph. However, constraint-based layouts often require individual constraints over node pairs and introduce a degree of separation between the constraint definition and the data domain that drives the constraint. To facilitate the specification of customized constraint layouts, we contribute \projectname, a domain-specific language for specifying high-level constraints relative to properties of the backing data. Users identify nodes sets based on data or graph properties and apply high-level constraints within each set of nodes. Applying constraints to node sets minimizes specification effort on the part of the user. Identifying sets by node properties and applying constraints to node sets facilitates reapplication of customized layouts across distinct graphs by separating the constraint generation from individual properties of the nodes. We demonstrate the simplicity and generalizability of \projectname~on a series of real-world examples from ecological networks, biological systems, and social networks.

  \begin{classification}
  \CCScat{Human-centered computing}{}{Visualization}{Graph drawings}
  \end{classification}
}

% More formal definition of 'custmoized graph layout' - often achieved with manual customization
% Definitely need to have crisp definition
% More targeted at end-user perception
