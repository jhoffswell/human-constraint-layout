%!TEX root = constraint-layout.tex
\newcommand{\paperabstract}{
Constraints enable flexible graph layout by combining the ease of
automatic layout with customizations for a particular domain.
However, constraint-based layout often requires many individual
constraints defined over specific nodes and node pairs.
In addition to the effort of writing and maintaining a large number
of similar constraints, such constraints are specific to the
particular graph and thus cannot generalize to other graphs in the same
domain. To facilitate the specification of customized and generalizable
constraint layouts, we contribute \projectname: a domain-specific language
for specifying high-level constraints relative to properties of the backing
data. Users identify node sets based on data or graph properties and apply
high-level constraints within each set. Applying constraints to
node sets rather than individual nodes reduces specification
effort and facilitates reapplication of customized layouts
across distinct graphs. We demonstrate the conciseness,
generalizability, and expressiveness of \projectname~on a series of
real-world examples from ecological networks, biological systems, and
social networks. \feedback{Zening}{I don't know how feasible it is, but if you have some 
rough quantitative measure of how much specification effort can be saved 
by SetCoLa, that would be really cool. Perhaps that measure can be the 
number of constraints a user has to explicitly construct?}
}
