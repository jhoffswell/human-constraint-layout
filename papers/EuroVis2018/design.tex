%!TEX root = constraint-layout.tex
\vspace{-1px}
\section{Design of \projectname}
\projectname is a domain-specific language for concise specification of 
constrained graph layouts. To provide a reusable specification 
without explicit reference to individual nodes and edges, \projectname
applies constraints to groups of nodes defined by shared attributes. 
The central abstraction in \projectname is a \textbf{set}. The 
simplest elements of a set are graph nodes; however, \projectname also
supports hierarchical composition, with nested sets as elements.

A \projectname specification consists of one or more \textbf{constraint definitions},
along with an optional set of guides (reference elements that serve as 
positional anchors). An example \projectname specification for a small tree 
layout is shown in Figure~\ref{fig:small-tree-example}a. Each constraint 
definition includes a \textbf{set definition} and \textbf{constraint application}.
\projectname provides several operators for defining
sets based on node attributes and structural relations in the graph.
The result of a set definition produces one or more sets, which 
can have either distinct or overlapping elements. Each constraint definition can 
define one or more constraints (e.g., for position, ordering, 
or alignment), which are applied to the nodes within each separate set created by the set definition. 
In other words, constraints created for a constraint definition with multiple sets are 
applied to nodes \emph{within} each individual set, not \emph{between} 
the sets in the constraint definition.

The \projectname compiler takes an input graph 
and specification, and produces a set of instance-level constraints for 
an existing constrained graph layout solver. In this work, we target 
Dwyer~et~al.'s WebCoLa library \cite{WebCoLa} in order to support
interactive, web-based layouts. The \projectname compiler generates one
or more WebCoLa constraints for each \projectname constraint and produces a
specification for WebCoLa (Figure~\ref{fig:small-tree-example}c).
 In the following sections, 
we discuss the design of \projectname including the process for 
specifying sets, the types of constraints currently supported in \projectname, 
and how such constraints are applied over node sets. 
Our implementation of \projectname is available at the link
{\small\url{https://github.com/jhoffswell/human-constraint-layout}}.

% \smallTreeExample
\smallTreeExampleWebCoLa

%%%%%%%%%%%%%%%%%%%%%%%%%%%%%%%%%%%%%%%%%%%%%%%%%%%%%%%%%%%%%%
%%%%%%%%%%%%%%%%%%%%%%%%%%%%%%%%%%%%%%%%%%%%%%%%%%%%%%%%%%%%%%
\vspace{-5px}
\subsection{Specifying Sets in \projectname}
We provide several operators for defining sets based on node attributes 
or structural properties, which include (1) \emph{partitioning} nodes 
into disjoint sets, (2) specifying (potentially overlapping) sets via
\emph{predicates}, (3) \emph{collecting} nodes into sets based on key expressions,
and (4) \emph{composing} previously defined sets. Each of these set
definitions produces one or more sets for the constraint definition. For
partitions, predicates, or collections, the constraint definition may 
designate a set from which the elements should be drawn; by default, the
elements are simply all graph nodes. For each subsection, we show a sample
\projectname constraint in the header.

When new sets are defined, each set propagates the consistent properties
of its elements as a single property of its own. For example, in
Figure \ref{fig:small-tree-example}a, when the user
partitions the nodes based on their \texttt{depth} (Line
\texttt{17}), each new set is given a \texttt{depth} property the value of which 
matches all the internal elements. This property can later be referenced 
in the \projectname specification when the sets are treated as elements 
(e.g., in the order constraint, Line~\texttt{23}).

%%%%%%%%%%%%%%%%%%%%%%%%%%%%%%%%%%%%%%%%%%%%%%%%%%%%%%%%%%%%%%
\subsubsection{Partitioning Nodes into Sets}
\constraint{Ex: \texttt{partition} by \texttt{depth}}
The partition operator creates a collection of disjoint sets based on
properties of the node
(e.g., Figure~\ref{fig:small-tree-example}a, Line~\texttt{17}).
Given $n$ nodes to partition, this operator can produce at most $O(n)$ 
sets. The user may further limit 
the number of sets included by providing specific property values to 
\emph{include} or \emph{exclude} when producing the sets.
For each node, we identify a key based on the node values for the partition
properties and create sets based on the key. The \emph{include} parameter allows
the user to identify particular node values to look for when partitioning
and includes only those values; the exclude property does the opposite. 
For example in Figure~\ref{fig:syphilis-spec}, Line~\texttt{5}, the user
partitions the nodes by the \texttt{group} property and ignores the nodes
for which the value of \texttt{group} is \emph{``other.''} This parition 
will produce three sets with the nodes separated by \texttt{group}, and the 
13 nodes with group \emph{``other''} will not appear in any of these sets.

%%%%%%%%%%%%%%%%%%%%%%%%%%%%%%%%%%%%%%%%%%%%%%%%%%%%%%%%%%%%%%
\subsubsection{Specifying Sets with Predicates}
\constraint{Ex: \texttt{node.depth == 2}}
For more flexibility in the definition of sets, users
can specify a concrete list of sets, each defined by an arbitrary boolean
expression. For each expression, each node is evaluated by the expression
to determine if it should be included in the set. The user may
refer to properties of the node using dot syntax; for example,
\texttt{node.depth} refers to the depth property. In 
Figure~\ref{fig:syphilis-spec},~Lines~\texttt{13},~\texttt{14},~and~\texttt{15},
we define three sets based on the \texttt{group} property of the nodes.
The set on Line \texttt{14} includes nodes that are in the group 
\emph{``younger white women''} and nodes in the group \emph{``other.''}
Users may optionally specify a \texttt{name} for the set, which may be 
referred to in subsequent composite set definitions. With this specification 
strategy, it is possible to create node sets that are not disjoint, 
and may thus lead to unsatisfiable constraints.

%%%%%%%%%%%%%%%%%%%%%%%%%%%%%%%%%%%%%%%%%%%%%%%%%%%%%%%%%%%%%%
\subsubsection{Collecting Nodes Using Keys}
\constraint{Ex: \texttt{collect} \texttt{[}\emph{``id(),'' ``id(neighbors)''}\texttt{]}}
To combine the flexibility of \emph{predicates} with the automation of 
\emph{partition}, users may specify sets as a union based on key expressions.
For each element on which the constraint definition is applied, each key expression is
evaluated to identify the nodes in that set.
For example, in Figure~\ref{fig:tlr4-spec},~Line~\texttt{34}, the 
user creates a constraint definition that applies only to nodes with type
\emph{``unknown''} (in this case, only one node). On Line~\texttt{35}, the user 
creates a set definition with two key expressions: one key expression identifies
the \texttt{\_id} of the node (e.g., \texttt{node.\_id}), and the other 
key expression identifies the \texttt{\_id}s of the neighboring nodes 
(e.g., \texttt{node.neighbors.extract("\_id")}). The one set produced
contains the element itself and all its neighbors. We also include several 
built-in properties for identifying structural relationships in the 
graph, which are described in Section~\ref{sec:built-in-properties}.

%%%%%%%%%%%%%%%%%%%%%%%%%%%%%%%%%%%%%%%%%%%%%%%%%%%%%%%%%%%%%%
\vspace{10px}
\subsubsection{Composing Previously Defined Sets}
\constraint{Ex: \texttt{[} \emph{``set1''}, \emph{``set2''} \texttt{]}}
Sets may also be defined as hierarchical
compositions of previously defined sets. For example, in
Figure~\ref{fig:small-tree-example}a the first constraint definition
(named ``layer'') produces three sets via
partition~(Line~\texttt{17}). The next constraint definition performs 
composition, referencing only the ``layer'' set~(Line~\texttt{22}): the result
is a single set that contains the three layer sets as elements.
For composition, users may
refer to any named entities previously defined in the specification (e.g.,
previous set definitions or named sets produced from \emph{predicates}).
In the current version of \projectname, we only support composition via
set union, though future work should explore the types of set definitions
that other composition strategies could enable.

%%%%%%%%%%%%%%%%%%%%%%%%%%%%%%%%%%%%%%%%%%%%%%%%%%%%%%%%%%%%%%
\subsection{Built-In Properties of the Graph Structure}
\label{sec:built-in-properties}

In addition to defining constraints relative to node properties, it may
also be important to define constraints on properties
of the graph structure.  We support this with a number of built-in accessors.
These properties are automatically computed and added to the graph 
specification only when they are used in one of the \projectname constraints. 
These properties are only computed if such a property does not
already exist on the nodes and are subject to a number of expectations
regarding the graph input; for graphs that do not meet these expectations,
users are shown a warning and required to compute the properties
themselves. In this section, we describe each built-in property and discuss 
the expectations for use.

% \jheer{Not clear under what these expectations are and under what circumstances a graph does not meet them.}

\begin{description}
\item[\texttt{\_id}] The node index in the graph specification. This
  property is always computed regardless of whether or not it is referenced by the user. The
  \texttt{\_id} is a unique identifier and is used to 
  convert the \projectname constraints to low-level WebCoLa constraints.
\item[\texttt{depth}] One more than the max depth of the node's parents. 
	Root nodes (any nodes with no edges for which the node is the target)
	have a depth of zero. The \texttt{depth} property is only computed for
	graphs that do not contain cycles.
\item[\texttt{sources}] The list of nodes that have edges for which the
	current node is the target (e.g., all parent nodes).
\item[\texttt{targets}] The list of nodes that have edges for which the
	current node is the source (e.g., all child nodes).
\item[\texttt{neighbors}] The list of nodes that have edges connected to
  the current node. This property is the union of the \texttt{sources} and
  \texttt{targets} properties. Neighbors may also take an optional value
  as input, which returns all nodes with a graph distance less than or
  equal to the specified value.
% \item[\texttt{firstchild(<property>)}] The child node of the current node 
	% with the smallest value for the defined \texttt{property}. We can similarly
	% enable \texttt{lastchild}, which returns the largest value for the defined 
	% \texttt{property}.
\item[\texttt{incoming}] The list of edges in which the current node is the
	target~(e.g., the edges connecting the current node to all sources).
\item[\texttt{outgoing}] The list of edges in which the current node is the
	source~(e.g., the edges connecting the current node to all targets).
\item[\texttt{edges}] The list of edges that contain the current node. This
	property is the union of the \texttt{incoming} and \texttt{outgoing} edges.
\item[\texttt{degree}] The number of \texttt{neighbors}.
\end{description}
\constraintsFigure

% \jheer{Is \texttt{\_id} required? That might confuse readers after we make claims about \projectname's reusability.}

In the original graph specification, the links are defined by a source
and target node. However, whether or not these links are
directed or undirected is up to the user in how they
are treated in the graph layout. For example, properties such as
\texttt{neighbors} and \texttt{edges} are more appropriate than \texttt{sources}
for undirected graphs. We selected this list of properties as common structural elements
applicable to a variety of layout specifications. For example, the \texttt{depth}
property is useful for producing hierarchical tree layouts and the 
\texttt{sources}, \texttt{targets}, and \texttt{neighbors} properties
depict the relationship between nodes as dictated by the graph edges. 
There are many other properties that could be useful for graph layouts that are not included 
here and this list could easily be extended in the future to include other
common properties.

In addition to the built-in properties, we include several operators for 
manipulating the resulting lists of elements: \texttt{length}, \texttt{reverse},
\texttt{contains}, \texttt{sort}, and \texttt{extract}. The function 
\texttt{length()} returns the length of the list, \texttt{reverse()} 
reverses the order of the list, and \texttt{contains(value)} determines if
the list contains the identified value. The user may also choose to \texttt{sort} 
the list based on a property of the elements or \texttt{extract} the value of each
element for a particular property. Values can also be extracted from
nodes or edges individually using the dot syntax at any point. The user may use standard
array access to extract elements from the list (e.g., the first element of
the list is \texttt{list[0]}).

%\jheer{How are sets/arrays be treated within predicates? For example, predicate functions such as \texttt{isNeighbor} or \texttt{isSource} might be more convenient or appropriate than working with lists of nodes?}

% \jheer{What if edges have weights or attributes? Can those be queried? Finally, might other higher-order operators be useful? For example, identifying a shortest path and aligning elements along that path? (Actually, that sounds like it might be pretty useful in some situations!) Probably all beyond the scope of this paper, but still fun to think about...}
