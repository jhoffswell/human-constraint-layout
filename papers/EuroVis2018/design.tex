%!TEX root = constraint-layout.tex
\section{Design Considerations}

\todo{maybe it would be helpful to lay out the main design considerations
  for the language. e.g. talk about existing graph layouts and requirements
  when selecting or designing layouts by hand that could motivate our
  choice of constraints. Or would it be fine to just have this in the
  design section when discussing the rationale?}

%%%%%%%%%%%%%%%%%%%%%%%%%%%%%%%%%%%%%%%%%%%%%%%%%%%%%%%%%%%%%%
%%%%%%%%%%%%%%%%%%%%%%%%%%%%%%%%%%%%%%%%%%%%%%%%%%%%%%%%%%%%%%
\section{Design of \projectname}
\projectname\ is a domain-specific language for the creation of highly
customized graph layouts based on properties of the nodes. Users identify
element sets from the graph and apply constraints to all elements in a
given set. These sets can be hierarchically composed to create a flexible,
nested layout. \projectname\ reduces specification effort and allows the
user to focus on the graph layout in terms of high-level properties of the
data rather on individual nodes. These layout specifications can also be
easily reused across graphs for which the nodes have the same properties,
which allows the user to apply a single layout design across multiple
graphs in the same domain. In this section, we discuss the design of
\projectname\ including the process for specifying sets and applying
constraints.

\smallTreeExample

The main contribution of \projectname\ is the language for specifying
high-level constraints. \projectname\ specifications are composed of four
components: \emph{nodes}, \emph{links}, \emph{guides}, and
\emph{collections}. An example \projectname\ specification is shown in
Figure \ref{fig:small-tree-example}. Properties of the graph are included
as properties of the nodes and links. The user can then define collections
which are defined as a list of sets and a list of high-level constraints
that apply to the elements in each set of the collection. Users identify
\emph{sets} based on graph properties; sets can contain either graph nodes
or previously defined collections. Guides are an optional component and may
be used to define top-level layout boundaries (defined with an x and/or y
position), which can be referred to in the constraint specification.

%%%%%%%%%%%%%%%%%%%%%%%%%%%%%%%%%%%%%%%%%%%%%%%%%%%%%%%%%%%%%%
%%%%%%%%%%%%%%%%%%%%%%%%%%%%%%%%%%%%%%%%%%%%%%%%%%%%%%%%%%%%%%
\subsection{Specifying Sets in \projectname}
Each collection in \projectname\ is defined as a list of sets, where sets
are lists of elements. The elements in a set can be either nodes or
previously created sets, thus enabling a hierarchical composition of sets
for more complex layouts. In this language, we support several strategies
for specifying sets of nodes: \emph{partitioning}, \emph{expressions},
\orange{\emph{relations}}, and \emph{compositions}.

\textbf{Partitioning.} Users can partition all graph nodes into sets by
identify a property (\todo{Partition by multiple properties?}) of the nodes
with which to group the nodes (see Figure \ref{fig:small-tree-example},
Line \texttt{17}). Nodes are then separated into disjoint sets based on the
partition property. The user may also specify lists of values that identify
which property values to \texttt{include} or \texttt{ignore} when
performing the partition. These properties allow the user to create
partitions where some nodes are excluded from the partition, and are
therefore not placed in any of the partition sets.

\textbf{Expressions.} For more flexibility in the definition of sets, users
can specify a concrete lists of sets to compute using arbitrary boolean
expressions. For each expression, each node is evaluated based on the user
defined expression to determine if it is included in the set. The user may
refer to properties of the node using dot syntax (e.g.,
\texttt{node.<property>}). Users may also optionally specify a
\texttt{name} for the set, which may be used by the composition set
specification. For this specification strategy, it is possible to create
node sets that are not disjoint, and may thus produce unsatisfiable
constraints. \todo{example figure}

\textbf{Relations.} For some sets, it may be useful to identify particular
relationships between nodes, which can be specified as a relation. For each
node, \todo{finish this section}

\contradictionExample

\textbf{Compositions.} Sets may also be formed through the hierarchical
composition of collections. For example, in Figure
\ref{fig:small-tree-example} we first create a collection of sets using
partition on Line \texttt{17}, named ``layer.'' On Line \texttt{22} we
create a new collection that contains the ``layer'' set. Line \texttt{28}
and \texttt{32} of Figure \ref{fig:contradiction-example} show the same
strategy, whereas Line \texttt{23} shows the shorthand composition for
producing the collection containing one set. For compositions, the user may
refer to any named entities previously defined in the specification (e.g.,
collections or named sets produced from \emph{expressions}).

\todo{``from'' property for set specification?}

When a named set is defined, the set element promotes the properties that
are equivalent for all elements in the set to a property of the set
element. For example, in Figure \ref{fig:small-tree-example}, when the user
defines a partition on the nodes based on the \texttt{depth} (Line
\texttt{17}), then each set element is given a \texttt{depth} property with
the value for that set. This property may now be referred to in other parts
of the \projectname specification (e.g., Line \texttt{23}).


%%%%%%%%%%%%%%%%%%%%%%%%%%%%%%%%%%%%%%%%%%%%%%%%%%%%%%%%%%%%%%
%%%%%%%%%%%%%%%%%%%%%%%%%%%%%%%%%%%%%%%%%%%%%%%%%%%%%%%%%%%%%%
\subsection{Applying Constraints in \projectname}
\label{sec:constraints}
Once a collection has been defined, users can apply layout constraints to
each of the sets in the collection. We identify four different types of
constraints that can be applied to set elements: \texttt{alignment},
\texttt{position}, \texttt{order}, \texttt{circle}, and \texttt{unit}. In
the following sections, we describe the behavior and properties of each of
these constraints.

% design section: rationale for why these are the right constructs (sufficiently expressive)
	% why are they user friendly?
	% hypotheses and reason hope/belief/evidence

%%%%%%%%%%%%%%%%%%%%%%%%%%%%%%%%%%%%%%%%%%%%%%%%%%%%%%%%%%%%%%
\subsubsection{Alignment Constraints}
\constraint{\texttt{align} \emph{x} \texttt{axis}} Alignment constraints
specify the alignment of elements along an axis (Figure
\ref{fig:small-tree-example}, Line \texttt{18}). The constraint must
include the \texttt{axis} (e.g., \texttt{x} or \texttt{y}) on which the
elements should be aligned and may optionally include the
\texttt{orientation} for the alignment, one of: \texttt{center},
\texttt{left}, \texttt{right}, \texttt{top}, or \texttt{bottom}. When
applied to nodes, this constraint ensures that all the nodes are aligned
along the \texttt{axis}. When applied to sets, this constraint produces an
alignment based on the centroid or boundary of the set elements.

\emph{Rationale.} Alignment is a common characteristic seen when producing
graphs by hand \orange{citation} or utilizing particular layout strategies
\orange{citation} and is therefore crucial to support in \projectname.

%%%%%%%%%%%%%%%%%%%%%%%%%%%%%%%%%%%%%%%%%%%%%%%%%%%%%%%%%%%%%%
\subsubsection{Position Constraints}
\constraint{\texttt{position} \emph{right} \texttt{of}
  \emph{``top\_guide''}} Position constraints specify rules for the layout
of nodes relative to globally identified elements (\orange{figure}). The
constraint must identify a global element (e.g., a guide) to position the
set elements relative to and an orientation for the position (e.g.,
\texttt{left}, \texttt{right}, \texttt{above}, or \texttt{below}). The user
may optionally provide a \texttt{gap} for how far apart the elements should
be from the position element.

\emph{Rationale.} Position constraints allow the user to provide
overarching constraints relative to global elements such as guides. These
can be used to constrain the overall size of the graph or to section off
different areas. \todo{citations}

%%%%%%%%%%%%%%%%%%%%%%%%%%%%%%%%%%%%%%%%%%%%%%%%%%%%%%%%%%%%%%
\subsubsection{Order Constraints}
\constraint{\texttt{order} \emph{y} \texttt{axis} \texttt{by}
  \emph{``depth''}} Order constraints enforce a sort order on the set
elements (Figure \ref{fig:small-tree-example}, Line \texttt{23}). The
constraint must include the \texttt{axis} (e.g., \texttt{x} or \texttt{y})
on which the elements should be ordered. The constraint must also specify
the node property \texttt{by} which the order is determined. The user can
optionally define an explicit list of values for a custom
\texttt{order}. The user may also optionally include a boolean of whether
or not to \texttt{reverse} the order. Users may finally specify an optional
\texttt{band} property for the constraint that determines a size for each
set region; without the \texttt{band} property, the first and last elements
in the order are free to move on either end, whereas the \texttt{band}
enforces them to fit within the same sized space as the intermediate
elements.

\emph{Rationale.} Unlike position constraints, order constraints allow the
user to specify relationships between different sets of nodes to specify
the overall node layout. These relationships are particular useful for
sorting and hierarchical layouts on the different sets of nodes.

%%%%%%%%%%%%%%%%%%%%%%%%%%%%%%%%%%%%%%%%%%%%%%%%%%%%%%%%%%%%%%
\subsubsection{Circle Constraints}
\constraint{\texttt{circle} \texttt{around} \emph{``center''}} Circle
constraints allow the user to specify a ring layout for a set of
elements. The constraint must include which element the set elements should
be positioned around or may specify ``center'' for a generic center point
to be used.

\emph{Rationale.} Circular structures are a common layout requirement of
existing strategies \orange{citation} and can be crucial for relating
properties of the underlying structure.

%%%%%%%%%%%%%%%%%%%%%%%%%%%%%%%%%%%%%%%%%%%%%%%%%%%%%%%%%%%%%%
\subsubsection{Unit Constraints}
\constraint{\texttt{unit}} Unit constraints specify additional layout
properties for elements in the set. The user may optionally specify a
boolean \texttt{enclose} that prevents the set elements from overlapping
with other sets (\orange{figure}). The user may optionally specify an
\texttt{attract} property as either \texttt{center}, \texttt{left},
\texttt{right}, \texttt{top}, or \texttt{bottom} that causes the nodes to
be drawn to the specified area of the set (\orange{figure}). An optional
\texttt{padding} property determines the minimum amount of spacing that
should exist between elements in the unit and finally an optional
\texttt{spacing} property determines whether the spacing is either
\texttt{free} or \texttt{fixed}.

\emph{Rationale.} The unit constraint allows the user to enforce more
control on how the nodes within a set behave outside of the general layout
constraints specified earlier. These constraints allow the user to apply
more nuanced behaviors between the nodes.

%%%%%%%%%%%%%%%%%%%%%%%%%%%%%%%%%%%%%%%%%%%%%%%%%%%%%%%%%%%%%%
\subsection{Unsatisfiable Constraints in \projectname}
For a \projectname\ specification, it is possible to create node sets that
are \emph{not} disjoint and may thus be easily susceptible to the
specification of unsatisfiable constraints. Furthermore, through various
combinations of set specifications and constraint applications, it is
possible for the user to specify unsatisfiable constraints in the
\projectname\ layout. An example of a layout with unsatisfiable constraints
is shown in Figure \ref{fig:contradiction-example}. This layout completes
two partitions of the nodes (once by \texttt{level} and once by
\texttt{rank}); these collections lifted and an order constraint is applied
to enforce a hierarchy between the sets. However, due to the properties of
the nodes, the node with \texttt{rank 10} and \texttt{level 3} faces a
contradiction in the layout. Figure \ref{fig:contradiction-example}c shows
the error on the nodes that have constraints unsatisfied by the layout
(where darker means more error).

\todo{Does it make sense to talk about this here, or should we provide this
  as an example that we talk about later? Or should we talk about it more
  generally without an example and discuss why this is
  allowed?}\\ \\ \\ \\ \\ \\ \\ % Line breaks because latex is mad for some reason
