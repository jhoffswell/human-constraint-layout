%!TEX root = constraint-layout.tex
\usepackage{soul}
\usepackage{color}

\definecolor{red}{RGB}{178,34,34}
\definecolor{orange}{rgb}{1, 0.8, 0.4}
\definecolor{lightgreen}{RGB}{121, 210, 121}
\definecolor{lightpurple}{RGB}{153, 102, 255}
\definecolor{gray}{RGB}{166, 166, 166}

%% Note: One of the following blocks of \newcommmands should be commented in to show/hide comments
% \newcommand{\cut}[1]{}
% \newcommand{\todo}[1]{}

%% Note: Comment this in to see all comments and unfinished text in the paper.
\newcommand{\todo}[1]{\textcolor{red}{[TODO] \emph{#1}}}
\newcommand{\cut}[1]{\textcolor{red}{\st{#1}}}
\newcommand{\orange}[1]{\textcolor{orange}{#1}}
\newcommand{\green}[1]{\textcolor{lightgreen}{#1}}
\newcommand{\purple}[1]{\textcolor{lightpurple}{#1}}
\newcommand{\gray}[1]{\textcolor{gray}{#1}}

%%%%%%%%%%%%%%%%%%%%%%%%%%%%%%%%%%%%%%%%%%%%%%%%%%%%%%%%%%%%%%%%%%%%%%%%

% Text colors based on the figure colors
% \definecolor{figuregreen}{RGB}{210,231,211} % light figure green
% \definecolor{figureblue}{RGB}{183,217,254} % light figure blue
% \definecolor{figurepurple}{RGB}{205,183,219} % light figure purple
% \definecolor{figuregreen}{RGB}{82,102,0} % dark figure green
% \definecolor{figureblue}{RGB}{39,66,102} % dark figure blue
% \definecolor{figurepurple}{RGB}{83,54,102} % dark figure purple
\definecolor{figuregreen}{RGB}{134,166,0} % medium figure green: brightness: 40->65
\definecolor{figureblue}{RGB}{63,108,166} % medium figure blue: brightness: 40->65
\definecolor{figurepurple}{RGB}{135,88,166} % medium figure purple: brightness 40->65
\newcommand{\figuregreen}[1]{\textcolor{figuregreen}{#1}}
\newcommand{\figureblue}[1]{\textcolor{figureblue}{#1}}
\newcommand{\figurepurple}[1]{\textcolor{figurepurple}{#1}}

%%%%%%%%%%%%%%%%%%%%%%%%%%%%%%%%%%%%%%%%%%%%%%%%%%%%%%%%%%%%%%%%%%%%%%%%

\newcommand{\constraintname}[1]{\texttt{#1}}
\newcommand{\setcolaproperty}[1]{\texttt{#1}}
\newcommand{\setcolavalue}[1]{\emph{#1}}
\newcommand{\setcolareference}[1]{``#1''}
\newcommand{\setname}[1]{``#1''}
\newcommand{\nodeproperty}[1]{\texttt{#1}}
\newcommand{\nodevalue}[1]{\emph{#1}}

%%%%%%%%%%%%%%%%%%%%%%%%%%%%%%%%%%%%%%%%%%%%%%%%%%%%%%%%%%%%%%%%%%%%%%%%

\newcommand{\linenumber}[1]{Line \texttt{#1}}
\newcommand{\figureline}[3]{Figure~\ref{#1}#2, \linenumber{#3}}
