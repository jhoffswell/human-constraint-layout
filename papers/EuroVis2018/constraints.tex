%!TEX root = constraint-layout.tex
\section{\projectname Constraints and WebCoLa Implementation}
\label{sec:constraints}
We identified and implemented seven constraints for \projectname. In 
this work, we target  Dwyer et al.'s WebCoLa library \cite{WebCoLa} in 
order to support interactive, web-based layouts. 

We identify four different types of
constraints that can be applied to set elements: \texttt{alignment},
\texttt{position}, \texttt{order}, \texttt{circle}, and \texttt{unit}. In
the following sections, we describe the behavior and properties of each of
these constraints.

In WebCoLa, constraints are defined based on the \texttt{\_id} of the
node in the graph specification. For our implementation, we start by
computing the node sets; once we have the sets, we can define the WebCoLa
constraints for the nodes in the set. We leverage two of WebCoLa's
constraint types for our implementation: \emph{alignment} constraints 
and \emph{position} constraints. For other \projectname\ constraints, we
approximate the behavior as described in the following sections.

For each constraint type, we define the number of WebCoLa constraints
generated per \projectname\ constraint. Let $s$ be the number of sets per 
and $n$ be the number of elements per set for each \projectname\ constraint.
Let $g$ be the number of guides defined in the \projectname\ specification.

%%%%%%%%%%%%%%%%%%%%%%%%%%%%%%%%%%%%%%%%%%%%%%%%%%%%%%%%%%%%%%
\subsection{Alignment Constraints}
\constraint{\texttt{align} \emph{x} \texttt{axis}} Alignment constraints
specify the alignment of elements along an axis (Figure
\ref{fig:small-tree-example}, Line \texttt{18}). The constraint must
include the \texttt{axis} (e.g., \texttt{x} or \texttt{y}) on which the
elements should be aligned and may optionally include the
\texttt{orientation} for the alignment, one of: \texttt{center},
\texttt{left}, \texttt{right}, \texttt{top}, or \texttt{bottom}. When
applied to nodes, this constraint ensures that all the nodes are aligned
along the \texttt{axis}. When applied to sets, this constraint produces an
alignment based on the centroid or boundary of the set elements.

\emph{Rationale.} Alignment is a common characteristic seen when producing
graphs by hand \orange{citation} or utilizing particular layout strategies
\orange{citation} and is therefore crucial to support in \projectname.

\jheer{Why only x/y axes? Why not polar coordinates, for example? Or along arbitrary line segments defined as guides? I'm not saying you *should* support these, only that it's not clear why you focus where you do. Also I don't totally understand what alignment means here (nor what the default would be if unspecified). Readers may be confused about what you mean by applied to nodes vs. applied to sets, as earlier we talk about sets as the units that constraints are applied to. Also, centroid or boundary, which is used when and why?}

\emph{Implementation.}
Alignment constraints are one of the constraint types supported by WebCoLa;
once the node sets have been identified, we can generate a single WebCoLa
alignment constraint for each set in each \projectname\ alignment constraint.

\todo{We don't really have alignment working on set elements, only on node elements.}
\todo{Orientation is not working on nodes yet, but shouldn't be too hard to get working}

%%%%%%%%%%%%%%%%%%%%%%%%%%%%%%%%%%%%%%%%%%%%%%%%%%%%%%%%%%%%%%
\subsection{Position Constraints}
\constraint{\texttt{position} \emph{right} \texttt{of}
  \emph{``top\_guide''}} Position constraints specify rules for the layout
of nodes relative to globally identified elements (\orange{figure}). The
constraint must identify a global element (e.g., a guide) to position the
set elements relative to and an orientation for the position (e.g.,
\texttt{left}, \texttt{right}, \texttt{above}, or \texttt{below}). The user
may optionally provide a \texttt{gap} for how far apart the elements should
be from the position element.

\emph{Rationale.} Position constraints allow the user to provide
overarching constraints relative to global elements such as guides. These
can be used to constrain the overall size of the graph or to section off
different areas. \todo{citations}

\jheer{Why only relative guides? Why can't relative position constraints be applied between sets?}

\emph{Implementation.}
Position constraints are also a built-in constraint type in WebCoLa.
Once we have defined the node sets, we generate a WebCoLa position constraint
for each node in the set relative to the global element (i.e., guide node)
that is used in the constraint. We create one WebCoLa constraint for each node
in each set, for each \projectname\ constraint.

\todo{position constraints on set elements}

%%%%%%%%%%%%%%%%%%%%%%%%%%%%%%%%%%%%%%%%%%%%%%%%%%%%%%%%%%%%%%
\subsection{Order Constraints}
\constraint{\texttt{order} \emph{y} \texttt{axis} \texttt{by}
  \emph{``depth''}} Order constraints enforce a sort order on the set
elements (Figure \ref{fig:small-tree-example}, Line \texttt{23}). The
constraint must include the \texttt{axis} (e.g., \texttt{x} or \texttt{y})
on which the elements should be ordered. The constraint must also specify
the node property \texttt{by} which the order is determined. The user can
optionally define an explicit list of values for a custom
\texttt{order}. The user may also optionally include a boolean of whether
or not to \texttt{reverse} the order. Users may finally specify an optional
\texttt{band} property for the constraint that determines a size for each
set region; without the \texttt{band} property, the first and last elements
in the order are free to move on either end, whereas the \texttt{band}
enforces them to fit within the same sized space as the intermediate
elements. \jheer{I don't totally follow what band is.}

\jheer{Similar question as above regarding x/y vs. other possible options.}

\emph{Rationale.} Unlike position constraints, order constraints allow the
user to specify relationships between different sets of nodes to specify
the overall node layout. These relationships are particular useful for
sorting and hierarchical layouts on the different sets of nodes.

\emph{Implementation}
\todo{Finalize description of order procedure}
\todo{order constraints on set elements}

%%%%%%%%%%%%%%%%%%%%%%%%%%%%%%%%%%%%%%%%%%%%%%%%%%%%%%%%%%%%%%
\subsection{Circle Constraints}
\constraint{\texttt{circle} \texttt{around} \emph{``center''}} Circle
constraints allow the user to specify a ring layout for a set of
elements. The constraint must include which element the set elements should
be positioned around or may specify ``center'' for a generic center point
to be used.

\jheer{I'm confused regarding the center. What is a generic center point? Can one specify a x,y point or related guide? What does it mean to put an element in the center? Does this constraint work with sets of sets?}

\emph{Rationale.} Circular structures are a common layout requirement of
existing strategies \orange{citation} and can be crucial for relating
properties of the underlying structure.

\emph{Implementation.}
Circle constraints are not a supported constraint type in WebCoLa. To 
demonstrate the utility of this constraint type we approximated the behavior
in our WebCoLa implementation. To do this, we first add a temporary node
to act as the center of the constraint. We then add a link between each 
node in the set and the temporary center. \todo{Finish...}

\todo{Confirm description of procedure}
\todo{I've looked into two ways of approximating the circle procedure. 
(1) Add edges between each node in the circle and the center, and around
the circle edge. (2) Produce the layout with position constraints and 
then convert to polar coordinates.}

%%%%%%%%%%%%%%%%%%%%%%%%%%%%%%%%%%%%%%%%%%%%%%%%%%%%%%%%%%%%%%
\subsection{Cluster Constraints}
\constraint{\texttt{cluster}} 
\todo{This section}

%%%%%%%%%%%%%%%%%%%%%%%%%%%%%%%%%%%%%%%%%%%%%%%%%%%%%%%%%%%%%%
\subsection{Hull Constraints}
\constraint{\texttt{hull}} 
\todo{This section}

%%%%%%%%%%%%%%%%%%%%%%%%%%%%%%%%%%%%%%%%%%%%%%%%%%%%%%%%%%%%%%
\subsection{Padding Constraints}
\constraint{\texttt{padding 5}} 
\todo{This section}

\subsection{Creating Guides in WebCoLa}
\constraint{$O(g)$ new nodes}
In \projectname, the user may define guides to control the graph layout.
To use these guides in WebCoLa constraints, we add a new node to the graph
for each guide and generate constraints relative to this node. These 
temporary nodes are included in WebCoLa's layout but are hidden from the
final visualization of the graph layout.
