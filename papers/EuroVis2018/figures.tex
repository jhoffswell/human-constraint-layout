%!TEX root = constraint-layout.tex

% \newcommand{\teaserFigure}{
% 	\teaser{
% 		\includegraphics[width=\linewidth]{figures/serengeti-layout.pdf}
% 		\centering
% 	  \caption{\label{fig:teaser}}

% 	}
% }

\newcommand{\serengetiLayoutColumn}{
  \begin{figure}[t!]
    \centering
    \includegraphics[width=0.81\columnwidth]{figures/serengeti-layout-column.pdf}
    {\caption{\label{fig:serengeti-layout}
    The layout for the Serengeti food web using our constraint language, as compared to Baskerville et al. \cite{baskerville2011spatial}.}}
    \vspace{-40px}
  \end{figure}
}

%%%%%%%%%%%%%%%%%%%%%%%%%%%%%%%%%%
%%%%%%%%% Demonstration %%%%%%%%%%
%%%%%%%%%%%%%%%%%%%%%%%%%%%%%%%%%%

\newcommand{\serengetiLayout}{
  \begin{figure*}[t]
    \centering
    \includegraphics[width=\textwidth]{figures/serengeti-layout.pdf}
    \vspace{-20px}
    {\caption{\label{fig:serengeti-layout}
    The layout for the Serengeti food web using our constraint language, as compared to Baskerville et al. \cite{baskerville2011spatial}. \todo{retake photos on retina screen} \todo{label the two sides of the figure a/b}}}
  \end{figure*}
}

\newcommand{\serengetiSpec}{
  \begin{figure}[t]
    \centering
    \includegraphics[width=\columnwidth]{figures/serengeti-spec.pdf}
    \vspace{-20px}
    {\caption{\label{fig:serengeti-spec}
    The \projectname~specification for the Serengeti food web shown in Figure~\ref{fig:serengeti-layout}. \orange{It might be cool to label the specification with the number of WebCoLa constraints produced for each high level constraint? Try playing with this a little.}}}
  \end{figure}
}