%!TEX root = constraint-layout.tex
\section{Related Work}
There are many areas of related work surrounding graph visualization; we
discuss general layout techniques that motivate this work and expand on the
discussion of domain specific layouts.

\subsection{General Graph Layout}
There are numerous approaches to graph layout that leverage structural
information from the graph to produce the layout \cite{herman2000graph,eades2010graph,gibson2013survey}.
For node-link diagrams, the Reingold \& Tilford layout~\cite{reingold1981tidier}
iteratively arranges graph nodes to produce compact, symmetrical tree layouts.
Radial layouts \orange{citation} similarly represent node-link diagrams but rather use polar coordinates,
with the root placed at the center point. Sugiyama-style layouts \cite{sugiyama1981methods} iteratively
arrange nodes into layers and ensure the arrangement visually depicts the hierarchy.
Force-directed techniques \cite{tutte1963draw,kobourov2012spring,quinn1979forced,fruchterman1991graph} 
run physical simulations that incorporate forces on the nodes and
may treat edges as springs meant to draw neighboring nodes
together and minimize the overall energy. 
There are a variety of tools that support graph drawing with different layout 
techniques, including D3.js~\cite{bostock:d3}, Gephi~\cite{bastian2009gephi},
Graphviz~\cite{ellson2001graphviz}, and Cytoscape~\cite{shannon2003cytoscape}.

To represent the hierarchy for trees in a more compact space, treemaps \cite{johnson1991tree}
shed the node-link metaphor and represent the hierarchy via enclosure. 
Icicle plots \cite{kruskal1983icicle} and sunburst diagrams \orange{citation} 
use space filling nodes and represent the hierarchy as a subdivision of the parent nodes.
Arc diagrams \orange{citation} and adjacency matrices \orange{citation}
emphasize the links in more complex networks but it can be harder to trace
paths through the network and the order of the nodes impacts the visual 
patterns.
Many of the previously mentioned techniques produce layouts based on structural
properties of the graph, but there is also a large field of 
related work surrounding graph layout techniques that emphasize particular
aesthetic properties. \textsc{hola} \cite{kieffer2016hola} produces orthogonal 
layouts similar to those produced by hand, informed by common aesthetic properties 
used in manual layouts.

\subsection{Constraint Layout Techniques}
Constraints are another strategy for creating more complex layouts that further
represent structural or aesthetic properties of the graph. IPSep-CoLa extends
the force-directed layout to apply separation constraints on pairs of nodes.
Dig-CoLa \cite{dwyer2005dig} encodes the hierarchy of nodes as constraints
and attempts to minimize the stress on the overall graph layout. 
Kieffer et al.~\cite{kieffer2013incremental} present a constraint-based 
layout for creating graphs with node and edge alignment,
and demonstrate the feasibility of such a technique by incorporating it into
the interactive graph layout system, Dunnart \cite{dwyer2008dunnart}.
WebCoLa~\cite{WebCoLa} is a JavaScript library for constraint-based layout 
that can be used alongside D3 or Cytoscape~\cite{shannon2003cytoscape}. 
These constraint layout techniques help to emphasize structural properties of
the graph such as hierarchy or containment. While WebCoLa can support 
customized constraints in addition to more general structural constraints,
the specification of individual inter-node constraints can be labor intensive.
\projectname aims to reduce the burden of specifying customized constraints
to enable the design of domain-specific and generalizable layouts.

%\jheer{The prose here feels more *tool* focused (e.g., Graphviz, D3, Cytoscape) then *technique* focused (e.g., different layout approaches). I think you should take a step back and think about the big picture here. What are the common automatic (non-constrained based) approaches to node-link diagram layout? What structural features does each show? What are their limitations? Then go on to discuss contrained and/or customized layouts and their value for domain-specific graph layout. The DiG-CoLa paper does a nice job of this, contrasting force-directed and Sugiyama approaches, and showing how DiG-CoLa in some ways enables hybrids of the two. Also, right now the critical discussion of constrained layouts -- their main concepts, strengths, and weaknesses -- is missing.}

\subsection{Domain Specific Graph Visualization}
Several techniques have been developed to utilize
domain specific information for graph layout, but these techniques do not cover
all possible domains of interest. In a blog post describing her approach to
visualizing food webs~\cite{kearney2016blog}, Kearney notes that the node
placement algorithm should \emph{``allow contraining y-position to match
  trophic level while allowing free movement in the x-direction. With no
  such algorithm seemingly readily available, I decided to create my
  own.''} Kearney has developed plugins for D3~\cite{kearney2017d3} and
Ecopath~\cite{kearney2017ecopath} to aid in the visualization of
food webs. Baskerville et al.\ visualized the Serengeti food web
\cite{baskerville2011spatial}; in addition to the static visualization
presented in the paper, they developed an interactive graph
\cite{baskerville2011interactive} created with D3 that approximates the
paper figure. Motivated by this related work and the
challenges Kearney describes, our constraint language aims to provide users
with a lightweight strategy for identifying domain specific constraints.

Biological systems are a common domain requiring customized visualizations.
Cerebral \cite{barsky2008cerebral} is a tool designed to visualize
biological systems and support interactive exploration of different
experimental conditions. Genc and Dogrusoz \cite{saraiya2005visualizing}
describe a constrained, force-directed layout technique for visualizing 
biological pathways. Cytoscape \cite{shannon2003cytoscape} is a
visualization system designed to explore biomolecular interaction networks
and provides a framework for accepting customized plugins to extend the
system, such as WebCoLa~\cite{WebCoLa}. Each of these strategies was designed to
address the needs of domain experts who were unable to find the necessary
support within existing tools. Our work aims to reduce the barrier to
creating customized graph layouts by providing a compact way to specify domain
specific layouts that leverage expert knowledge.

For social networks, many visualizations leverage force-directed layouts
to demonstrate the connectedness or clustering of the nodes. However, some
visualizations may introduce additional separation or grouping to the layout
to highlight other properties such as ethnographic groupings \cite{rothenberg1998using},
the timeline of disease exposure \cite{fitzpatrick2001preventable,mcelroy2003network},
or differences in the types of relationships reported \cite{fu2011hiv}.

For each of these domains, the visualizations are created using specially
designed tools or layout algorithms to leverage properties of the data
specific to the domain. \projectname aims to support the design of such
customized layouts concisely and enable reuse of the layout across different
data sets in the domain.

% alan's casssowary
% other strategies for writing constriants (constraint based section)

% michael mcguffin - interactive graph modification - genealogical graphs
% david auber - tooklkit tulip 
% frank van ham - perceptual organization of graphs - layout graphs by hand (512)
	% automated don't match humans
% daniel archambeault - graph drawing

% stephen north - graph vis
% bell labs folks - energy minimization, stress majorization

% FOCUS ON the dsl angle - mismatch between user can specify and the
% implementation as constraint program
