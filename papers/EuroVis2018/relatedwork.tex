%!TEX root = constraint-layout.tex
\vspace{-5px}
\section{Related Work}
A great deal of prior work has contributed to graph visualization.
We discuss general layout techniques and constraint approaches
that motivate this work, then expand upon domain-specific layouts.

\vspace{-5px}
\subsection{General Graph Layout}
Graph layout approaches often leverage the underlying structure to produce the 
layout \cite{herman2000graph,eades2010graph,gibson2013survey}.
For node-link diagrams of hierarchical data, Reingold \& Tilford's ``tidy'' 
layout~\cite{reingold1981tidier} arranges graph nodes into compact, 
symmetrical tree layouts based on aesthetic properties.
Radial layouts \cite{battista1998graph,herman2000graph} follow similar procedures using polar 
coordinates, with a root node placed at the origin.
Sugiyama-style layouts~\cite{sugiyama1981methods} visualize directed graphs by first assigning nodes to hierarchical layers and then iteratively adjusting node placement to minimize edge crossings.
Force-directed techniques~\cite{tutte1963draw,kobourov2012spring,quinn1979forced,fruchterman1991graph} 
use physical simulation and/or optimization methods that model repulsive forces between nodes and spring-like forces on edges, and attempt to minimize the overall energy. 
A number of popular tools support graph drawing, including D3.js~\cite{bostock:d3}, Gephi~\cite{bastian2009gephi},
Graphviz~\cite{ellson2001graphviz}, and Cytoscape~\cite{shannon2003cytoscape}.

\subsection{Constraint-Based Layout Techniques}
Extending an existing layout method to support constraints enables customized 
layouts that emphasize important structural or aesthetic properties of the graph. 
Dig-CoLa~\cite{dwyer2005dig} encodes the hierarchy of nodes as constraints
and attempts to minimize the overall stress; this technique
is a hybrid strategy that combines automatic hierarchical 
layout with undirected layouts to ensure downward pointing edges. 
\mbox{IPSep-CoLa~\cite{dwyer2006ipsep}} extends
force-directed layouts to apply separation constraints 
on pairs of nodes to support properties such as customized node ordering or
downward pointing edges.
Dwyer and Robertson~\cite{dwyer2009layout} present a strategy for supporting 
non-linear constraints (such as circle constraints) that does not constrain node
positions along a single axis, but can incorporate these techniques within
alternative layout strategies. 

Kieffer et al.~\cite{kieffer2013incremental} present a force-directed,
constraint-based layout for creating graphs with node and edge alignment, 
and demonstrate its effectiveness for interactive refinement within
the interactive graph layout system Dunnart~\cite{dwyer2008dunnart}.
Dwyer and Wybrow developed libcola \cite{libcola}, which utilizes constraints within a
force-directed graph layout \cite{dwyer2008topology} using 
stress majorization. Stress majorization~\cite{gansner2004graph} 
is a technique used for graph layout that has been extended for efficient 
application on constrained layouts \cite{dwyer2007constrained,wang2018revisiting}.
Compound constraints in libcola allow the application of constraints to a list
of nodes in the graph and may further introduce dummy nodes for the layout.
SetCoLa similarly supports this behavior with constraints defined over sets
of graph nodes and further enables hierarhical composition of sets to support
expressive, nested layouts.

WebCoLa~\cite{WebCoLa} is a JavaScript library based on libcola 
for constraint-based layout in a web-programming context that can be used alongside
D3~\cite{bostock:d3} or Cytoscape~\cite{shannon2003cytoscape}.
WebCoLa enables constraints on the alignment and
position of nodes as well as high-level properties such as flow (to ensure
edges point in the same direction) and non-overlapping constraints.
While WebCoLa can support customized constraints,
the specification of individual inter-node constraints can be labor intensive.
\projectname aims to reduce the burden of specifying customized constraints
to enable the design of domain-specific and generalizable layouts.

\vspace{-5px}
\subsection{Domain-Specific Graph Visualization}
Several techniques have been developed to reflect domain-specific concerns
within graph layouts. However, these techniques
tend to be highly-specialized, and so may not apply to other possible
domains of interest. For example, ecological networks are a common visualization
to include in publications to show the relationships amongst organisms in
an ecosystem. Baskerville~et~al.\ produced a customized visualization of 
the Serengeti food web~\cite{baskerville2011spatial} in which the nodes
are positioned based on their trophic level (e.g., the role of the
organism within the larger food chain) and further grouped based on a
Bayesian classification of the elements;  in addition to the 
static visualization, Baskerville~et~al.\ 
published an interactive version of the graph online~\cite{baskerville2011interactive}. 

Despite frequently publishing visualizations of oceanic food 
webs~\cite{kearney2012coupling,kearney2013amplification},
Kearney describes several challenges around the design of such visualizations
in a blog post~\cite{kearney2016blog}; Kearney notes that the node
placement algorithm should \emph{``allow constraining y-position to match
  trophic level while allowing free movement in the x-direction. With no
  such algorithm seemingly readily available, I decided to create my
  own.''} Kearney developed plugins for D3~\cite{kearney2017d3} and
Ecopath~\cite{kearney2017ecopath} to visualize food webs. 
Motivated by challenges such as those
described by Kearney, \projectname aims to provide users
with a lightweight means for authoring domain-specific constraints for customized layouts.

Biological systems also often require customized visualizations.
Cerebral~\cite{barsky2008cerebral} is designed to visualize
biological systems and supports interactive exploration of different
experimental conditions. Genc and Dogrusoz~\cite{genc2003constrained}
describe a constrained force-directed layout technique for visualizing 
biological pathways. Cytoscape~\cite{shannon2003cytoscape} is a
visualization system designed to explore biomolecular interaction networks
and provides a framework for accepting customized plugins, including 
a WebCoLa~\cite{WebCoLa} plugin for constraint-based layouts.
CrowdLayout~\cite{singh2018crowdlayout} introduces a strategy for crowdsourcing 
biological network layouts from novices that produces more high-quality
layouts than Cerebral~\cite{barsky2008cerebral} or Graphviz~\cite{ellson2001graphviz}.

Kieffer et al.'s work on incremental grid layouts~\cite{kieffer2013incremental}
was motivated by related work for grid layouts of biological 
networks~\cite{barsky2008cerebral,kojima2007efficient,li2005grid}, 
but aims to provide a more flexible mechanism for creating the constraints 
by supporting SBGN (Systems Biology Graphical Notation). In later
work, Kieffer et al.~\cite{kieffer2016hola} improve upon grid layout techniques
by first identifying the aesthetic criteria humans use for manual graph layout,
then producing a new algorithm named HOLA, which employs these techniques for improved,
human-like layouts.

Social networks often leverage force-directed layout techniques
to demonstrate the connectedness or clustering of the nodes. However, some
layouts may introduce additional separation or clustering
to highlight properties such as ethnographically-identified groups~\cite{rothenberg1998using},
the timeline of disease
exposure~\cite{fitzpatrick2001preventable,mcelroy2003network},
or differences in reported relationship types~\cite{fu2011hiv}.
For each of these domain areas, the visualizations are created using specially
designed tools or layout algorithms to leverage properties of the data
specific to the domain. With \projectname, we aim to reduce the barrier to
creating customized graph layouts by providing a compact way to specify
reusable domain-specific layouts that incorporate expert knowledge.


% alan's casssowary
% other strategies for writing constriants (constraint based section)

% michael mcguffin - interactive graph modification - genealogical graphs
% david auber - tooklkit tulip 
% frank van ham - perceptual organization of graphs - layout graphs by hand (512)
	% automated don't match humans
% daniel archambeault - graph drawing

% stephen north - graph vis
% bell labs folks - energy minimization, stress majorization

% FOCUS ON the dsl angle - mismatch between user can specify and the
% implementation as constraint program
