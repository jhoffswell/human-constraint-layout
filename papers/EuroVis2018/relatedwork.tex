%!TEX root = constraint-layout.tex
\section{Related Work}

A great deal of prior work has contributed to graph visualization.
We discuss general layout techniques and constraint approaches
that motivate this work, then expand upon domain-specific layouts.

\subsection{General Graph Layout}
Many graph layout approaches leverage the graph structure to produce the 
layout \cite{herman2000graph,eades2010graph,gibson2013survey}.
For node-link diagrams of hierarchical data, Reingold \& Tilford's ``tidy'' 
layout~\cite{reingold1981tidier} iteratively arranges graph nodes to
produce compact, symmetrical tree layouts.
Radial layouts \cite{battista1998graph,herman2000graph} follow similar procedures but use polar 
coordinates, with a root node placed at the origin.
Sugiyama-style layouts~\cite{sugiyama1981methods} visualize directed graphs by first assigning nodes to hierarchical layers and then iteratively adjusting node placement to minimize edge crossings.
Force-directed techniques~\cite{tutte1963draw,kobourov2012spring,quinn1979forced,fruchterman1991graph} 
use physical simulation and/or optimization methods that model repulsive forces between nodes and spring-like forces on edges, and then attempt to minimize the overall energy. 
A number of popular tools support graph drawing with various layout 
techniques, including D3.js~\cite{bostock:d3}, Gephi~\cite{bastian2009gephi},
Graphviz~\cite{ellson2001graphviz}, and Cytoscape~\cite{shannon2003cytoscape}.

\subsection{Constraint-Based Layout Techniques}
Extending an existing layout method to support constraints enables
customized layouts that may emphasize important
structural or aesthetic properties of the graph. IPSep-CoLa~\cite{dwyer2006ipsep} extends
force-directed layout to apply separation constraints on pairs of nodes.
This strategy allows the user to introduce constraints to enforce properties 
such as downward pointing edges or ordering based on node properties.
Dig-CoLa~\cite{dwyer2005dig} encodes the hierarchy of nodes as constraints
and attempts to minimize the stress on the overall graph layout; this strategy
combines automatic hierarchical layout with undirected layouts for a hybrid
visualization strategy that can ensure downward pointing edges to more
effectively represent the graph hierarchy. 
WebCoLa~\cite{WebCoLa} is a JavaScript library for constraint-based layout 
that can be used alongside D3~\cite{bostock:d3} or Cytoscape~\cite{shannon2003cytoscape}.
WebCoLa supports the specification of constraints on the alignment and relative
position of nodes as well as high-level properties such as flow (to ensure
edges point in the same direction) and non-overlapping constraints.
Kieffer et al.~\cite{kieffer2013incremental} present a force-directed,
constraint-based layout for creating graphs with node and edge alignment.
Kieffer et al. show that this strategy is effective even for interactive refinement 
by incorporating it into the interactive graph layout system Dunnart~\cite{dwyer2008dunnart}. 
These constraint-based layout techniques are often used to emphasize structural properties of
the graph such as hierarchy or containment. While WebCoLa can support 
customized constraints in addition to more general structural constraints,
the specification of individual inter-node constraints can be labor intensive.
\projectname aims to reduce the burden of specifying customized constraints
to enable the design of domain-specific and generalizable layouts.

\subsection{Domain-Specific Graph Visualization}
Several techniques have been developed to reflect domain-specific concerns
within graph layouts. However, these techniques
tend to be highly-specialized, and so may not apply to other possible
domains of interest. For example, ecological networks are a common visualization
to include in publications to show the relationships amongst organisms in
an ecosystem. Baskerville et al.\ produced a customized visualization of 
the Serengeti food web~\cite{baskerville2011spatial};  in addition to the 
static visualization presented in their paper, Baskerville et al.\ 
published an interactive version online~\cite{baskerville2011interactive}. 
Despite frequently publishing food web visualizations \cite{kearney2012coupling,kearney2013amplification},
Kearney describes several challenges around the design of such visualizations
in a blog post~\cite{kearney2016blog}; Kearney notes that the node
placement algorithm should \emph{``allow constraining y-position to match
  trophic level while allowing free movement in the x-direction. With no
  such algorithm seemingly readily available, I decided to create my
  own.''} Kearney developed plugins for D3~\cite{kearney2017d3} and
Ecopath~\cite{kearney2017ecopath} to produce food webs visualizations. 
Motivated by this related work and the
challenges described by Kearney, our constraint language aims to provide users
with a lightweight means for authoring domain-specific constraints.

Biological systems are a common domain requiring customized visualizations.
Cerebral~\cite{barsky2008cerebral} is a tool designed to visualize
biological systems and support interactive exploration of different
experimental conditions. Kieffer et al.'s work on incremental grid layouts 
was motivated by related work  for grid layouts of biological 
networks~\cite{barsky2008cerebral,kojima2007efficient,li2005grid}, 
but aims to provide a more flexible mechanism for creating the constraints 
by supporting SBGN (Systems Biology Graphical Notation). In later
work, Kieffer et al.~\cite{kieffer2016hola} improve upon grid layout techniques
by first identifying the aesthetic criteria humans use for manual graph layout,
then producing a new algorithm that employs these techniques for improved layouts.
Genc and Dogrusoz~\cite{genc2003constrained}
describe a constrained force-directed layout technique for visualizing 
biological pathways. Cytoscape~\cite{shannon2003cytoscape} is a
visualization system designed to explore biomolecular interaction networks
and provides a framework for accepting customized plugins to extend the
system. Each of these strategies was designed to
address the needs of domain experts who did not find necessary
support within existing tools.

For social networks, many visualizations leverage force-directed layouts
to demonstrate the connectedness or clustering of the nodes. However, some
layouts may introduce additional separation or clustering
to highlight properties such as ethnographically-identified
groups~\cite{rothenberg1998using},
the timeline of disease
exposure~\cite{fitzpatrick2001preventable,mcelroy2003network},
or differences in reported relationship types~\cite{fu2011hiv}.
For each of these domain areas, the visualizations are created using specially
designed tools or layout algorithms to leverage properties of the data
specific to the domain. With \projectname, we aim to reduce the barrier to
creating customized graph layouts by providing a compact way to specify
reusable domain-specific layouts that incorporate expert knowledge.


% alan's casssowary
% other strategies for writing constriants (constraint based section)

% michael mcguffin - interactive graph modification - genealogical graphs
% david auber - tooklkit tulip 
% frank van ham - perceptual organization of graphs - layout graphs by hand (512)
	% automated don't match humans
% daniel archambeault - graph drawing

% stephen north - graph vis
% bell labs folks - energy minimization, stress majorization

% FOCUS ON the dsl angle - mismatch between user can specify and the
% implementation as constraint program
