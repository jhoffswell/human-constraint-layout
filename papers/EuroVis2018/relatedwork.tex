%!TEX root = constraint-layout.tex
\section{Related Work}
There are many areas of related work surrounding graph visualization; we
discuss general layout techniques that motivate this work and expand on the
discussion of domain specific layouts.

\subsection{General Graph Layout}
Graph visualization has been a long-standing area of research and there are
several common layout techniques for visualizing graph data including tree
layouts or force-directed layouts
\cite{herman2000graph,eades2010graph}. Graphviz~\cite{ellson2001graphviz}
and Gephi~\cite{bastian2009gephi} are two examples of visualization engines
specific to graph layout. D3.js~\cite{bostock:d3} is a JavaScript library
that provides a number of built in layouts for graph data including
force-directed and hierarchical layouts. WebCoLa~\cite{WebCoLa} is a
JavaScript library for constraint-based layout that can be used alongside
D3 or Cytoscape~\cite{shannon2003cytoscape}. These techniques emphasize
certain structural properties, but are agnostic to the graph domain. While
WebCoLa can support some customized constraints in addition to more general
structure constraints, the specification of individual inter-node
constraints can be labor-intensive and require additional programming
expertise. Our language aims to reduce the burden of specifying constraints
to produce generalizable layouts.

In addition to these standard layout techniques, there is a large field of
related work surrounding graph layout techniques that emphasize particular
aesthetic or structural properties in the graph. \textsc{hola}
\cite{kieffer2016hola} presents a layout engine to produce layouts similar
to those produced by hand by identifying common aesthetic properties of the
graph used in manual layouts and using the results to drive the design of a
new layout algorithm. Kieffer et al. \cite{kieffer2013incremental} present
a constraint-based layout for creating graphs with node and edge alignment,
and demonstrate the feasibility of such a technique in an interactive
system, Dunnart \cite{dwyer2008dunnart}. These customized layouts reflect
general properties of the graph rather than domain specific details from
the data itself. Our work focuses on providing a general language for
supporting customized graph layouts based on domain specific data
properties with minimal programming expertise.

\todo{Add a specific paragraph/section talking about constraint based techniques}
% Include other resources like IpSepCoLa, DigCoLa, etc. in this new section

\subsection{Domain Specific Graph Visualization}
\todo{Shift from tool to technique for some of these, cerebral is the most
  tool like of any of them} Several tools have been developed to utilize
domain specific information for graph layout, but these tools do not cover
all possible domains of interest. In a blog post describing her approach to
visualizing food webs~\cite{kearney2016blog}, Kearney notes that the node
placement algorithm should \emph{``allow contraining y-position to match
  trophic level while allowing free movement in the x-direction. With no
  such algorithm seemingly readily available, I decided to create my
  own.''} Kearney has developed D3~\cite{kearney2017d3} and
Ecopath~\cite{kearney2017ecopath} plugins to aid in the visualization of
food webs. Baskerville et al. \cite{baskerville2011spatial} visualized the
Serengeti food web and included an interactive graph
\cite{baskerville2011interactive} created with D3 that approximates the
layout shown in their paper. Motivated by this related work and the
challenges Kearney describes, our constraint language aims to provide users
with a lightweight strategy for identifying domain specific constraints.

Cerebral \cite{barsky2008cerebral} is a tool designed to visualize
biological systems and support interactive exploration of different
experimental conditions. Cytoscape \cite{shannon2003cytoscape} is a
visualization system designed to explore biomolecular interaction networks
and provides a framework for accepting customized plugins to extend the
system, such as WebCoLa~\cite{WebCoLa}. Each of these tools was designed to
address the needs of domain experts who were unable to find the necessary
support within existing tools. Our work aims to reduce the barrier to
creating customized systems by providing a compact way to specify domain
specific graph layouts that leverage expert knowledge with minimal
programming expertise.

\orange{We will need to extend this section once we've worked out the third example}

% alan's casssowary
% other strategies for writing constriants (constraint based section)

% michael mcguffin - interactive graph modification - genealogical graphs
% david auber - tooklkit tulip 
% frank van ham - perceptual organization of graphs - layout graphs by hand (512)
	% automated don't match humans
% daniel archambeault - graph drawing

% stephen north - graph vis
% bell labs folks - energy minimization, stress majorization

% FOCUS ON the dsl angle - mismatch between user can specify and the
% implementation as constraint program
