%!TEX root = constraint-layout.tex
\section{Related Work}

A great deal of prior work has contributed to graph visualization.
We discuss general layout techniques that motivate this work and
then expand upon domain-specific layouts.

\subsection{General Graph Layout}
There are numerous approaches to graph layout that leverage structural
information from the graph to produce the layout \cite{herman2000graph,eades2010graph,gibson2013survey}.
For node-link diagrams of hierarchical data, Reingold \& Tilford's ``tidy'' 
layout~\cite{reingold1981tidier} iteratively arranges graph nodes to
produce compact, symmetrical tree layouts.
Radial layouts \orange{citation} follow similar procedures but using polar coordinates, with a root node placed at the origin.
Sugiyama-style layouts~\cite{sugiyama1981methods} visualize directed graphs by first assigning nodes to hierarchical layers and then iteratively adjusting node placement to minimize edge crossings.
Force-directed techniques~\cite{tutte1963draw,kobourov2012spring,quinn1979forced,fruchterman1991graph} 
use physical simulation and/or optimization methods that model repulsive forces between nodes and spring-like forces on edges, and then attempt to minimize the overall energy. 
A number of popular tools support graph drawing with various layout 
techniques, including D3.js~\cite{bostock:d3}, Gephi~\cite{bastian2009gephi},
Graphviz~\cite{ellson2001graphviz}, and Cytoscape~\cite{shannon2003cytoscape}.

Many of these techniques produce layouts based on structural
properties of the graph, but there is also a large field of 
related work surrounding graph layout techniques that emphasize particular
aesthetic properties. \textsc{hola} \cite{kieffer2016hola} produces orthogonal 
layouts similar to those produced by hand, informed by common aesthetic properties 
used in manual layouts.

\jheer{Hola is (to my mind) in a totally different category and should be cited / discussed, and perhaps related to issues of domain-specificity. For example, what domains are orthogonal layouts important for?}

\subsection{Constraint-Based Layout Techniques}
Extending an existing layout method to support \emph{constraints} enables customized layouts that may emphasize important
structural or aesthetic properties of the graph. IPSep-CoLa~\cite{dwyer2006ipsep} extends
force-directed layout to apply separation constraints on pairs of nodes.
Dig-CoLa~\cite{dwyer2005dig} encodes the hierarchy of nodes as constraints
and attempts to minimize the stress on the overall graph layout. 
Kieffer et al.~\cite{kieffer2013incremental} present a constraint-based 
layout for creating graphs with node and edge alignment,
and demonstrate the feasibility of such a technique by incorporating it into
the interactive graph layout system Dunnart~\cite{dwyer2008dunnart}.
WebCoLa~\cite{WebCoLa} is a JavaScript library for constraint-based layout 
that can be used with D3~\cite{bostock:d3} or Cytoscape~\cite{shannon2003cytoscape}. 
These constraint-based layout techniques are often used to emphasize structural properties of
the graph such as hierarchy or containment. While WebCoLa can support 
customized constraints in addition to more general structural constraints,
the specification of individual inter-node constraints can be labor intensive.
\projectname aims to reduce the burden of specifying customized constraints
to enable the design of domain-specific and generalizable layouts.

\jheer{Beyond listing lots of projects, perhaps we should say more about the nature of the constraints. What kinds of constraints do these tools support? That kind of overview is probably more important than listing all the various projects, and helps setup a later discussion of constraints supported by \projectname}

%\jheer{The prose here feels more *tool* focused (e.g., Graphviz, D3, Cytoscape) then *technique* focused (e.g., different layout approaches). I think you should take a step back and think about the big picture here. What are the common automatic (non-constrained based) approaches to node-link diagram layout? What structural features does each show? What are their limitations? Then go on to discuss contrained and/or customized layouts and their value for domain-specific graph layout. The DiG-CoLa paper does a nice job of this, contrasting force-directed and Sugiyama approaches, and showing how DiG-CoLa in some ways enables hybrids of the two. Also, right now the critical discussion of constrained layouts -- their main concepts, strengths, and weaknesses -- is missing.}

\subsection{Domain-Specific Graph Visualization}
Several techniques have been developed to reflect
domain-specific concenrs within graph layouts. However, these techniques
tend to be highly-specialized, and so may not apply to other possible
domains of interest. In a blog post describing her approach to
visualizing food webs~\cite{kearney2016blog}, Kearney notes that the node
placement algorithm should \emph{``allow constraining y-position to match
  trophic level while allowing free movement in the x-direction. With no
  such algorithm seemingly readily available, I decided to create my
  own.''} Kearney developed plugins for D3~\cite{kearney2017d3} and
Ecopath~\cite{kearney2017ecopath} to aid in the visualization of
food webs. Baskerville et al.\ visualized the Serengeti food web~\cite{baskerville2011spatial}; in addition to the static visualization
presented in their paper, they published an interactive version online~\cite{baskerville2011interactive}. Motivated by this related work and the
challenges described by Kearney, our constraint language aims to provide users
with a lightweight means for authoring domain-specific constraints.

Biological systems are a common domain requiring customized visualizations.
Cerebral~\cite{barsky2008cerebral} is a tool designed to visualize
biological systems and support interactive exploration of different
experimental conditions. Genc and Dogrusoz~\cite{genc2003constrained}
describe a constrained force-directed layout technique for visualizing 
biological pathways. Cytoscape~\cite{shannon2003cytoscape} is a
visualization system designed to explore biomolecular interaction networks
that provides a framework for accepting customized plugins to extend the
system, such as WebCoLa~\cite{WebCoLa}. Each of these strategies was designed to
address the needs of domain experts who did not find necessary
support within existing tools.

For social networks, many visualizations leverage force-directed layouts
to demonstrate the connectedness or clustering of the nodes. However, some
layouts may introduce additional separation or clustering
to highlight properties such as ethnographically-identified
groups~\cite{rothenberg1998using},
the timeline of disease
exposure~\cite{fitzpatrick2001preventable,mcelroy2003network},
or differences in reported relationship types~\cite{fu2011hiv}.

For each of these areas, the visualizations are created using specially
designed tools or layout algorithms to leverage properties of the data
specific to the domain. With \projectname, we aim to reduce the barrier to
creating customized graph layouts by providing a compact way to specify
reusable domain specific layouts that incorporate expert knowledge.


% alan's casssowary
% other strategies for writing constriants (constraint based section)

% michael mcguffin - interactive graph modification - genealogical graphs
% david auber - tooklkit tulip 
% frank van ham - perceptual organization of graphs - layout graphs by hand (512)
	% automated don't match humans
% daniel archambeault - graph drawing

% stephen north - graph vis
% bell labs folks - energy minimization, stress majorization

% FOCUS ON the dsl angle - mismatch between user can specify and the
% implementation as constraint program
