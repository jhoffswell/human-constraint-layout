%!TEX root = constraint-layout.tex
\section{Implementation of \projectname~using WebCoLa}
We created an implementation of \projectname~in which our high-level
constraints are used to generate node specific constraints in
WebCoLa~\cite{WebCoLa}. WebCoLa is an open-source JavaScript library for
creating high-quality, stable constraint layouts, and was therefore a
useful back end to demonstrate the utility of \projectname~on a number of
examples (see Section \ref{sec:examples}). In this section, we discuss the
constraint generation process for producing WebCoLa constraints.

\todo{Describe the constraint generation procedure for within and between
  sets as well as the number of constraints that each one generates.}

In addition to applying constraints relative to domain-specific properties,
it may be important to identify properties of the graph structure. For this
implementation, we support a number of built-in properties for the nodes
which are automatically computed and added to the graph specification when
used in the \projectname~constraints, prior to computing the WebCoLa
constraints. These properties are only computed if such a property does not
already exist on the nodes and are subject to a number of expectations
regarding the graph input; for graphs that do not meet these expectations,
the user is shown a warning and required to compute the properties
themselves.

\begin{description}
\item[\texttt{\_id}] The node index in the graph specification. This
  property is always computed regardless of whether or not it is used.
\item[\texttt{depth}] One more than the max depth of the node's
  parents. This property is only computed for graphs that contain only one
  root node and do not contain cycles.
\item[\texttt{parent}] The parent of the current node. This property is
  only allowed if the node has one parent; otherwise, it is defined as the
  parent with the smallest \texttt{\_id}.
\item[\texttt{firstchild}] The child node of the current node with the smallest \texttt{\_id}.
\item[\texttt{incoming}] The list of nodes that have edges where the
  current node is the target (e.g., all parent nodes).
\item[\texttt{outgoing}] The list of nodes that have edges where the
  current node is the source (e.g., all child nodes).
\item[\texttt{neighbors}] The list of nodes that have edges connected to
  the current node. This property is the join of the \texttt{incoming} and
  \texttt{outgoing} properties.
\item[\texttt{degree}] The number of \texttt{neighbors}.
\end{description}

\orange{confirm all descriptions}
\orange{Are there others? This is presumably quite extensible in the implementation.}
