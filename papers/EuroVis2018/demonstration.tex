%!TEX root = constraint-layout.tex
\section{Examples Reproduced in \projectname}
\label{sec:examples}

To demonstrate the simplicity and generalizability of \projectname, we
reproduce several real-world examples from ecological
networks~\cite{baskerville2011spatial}, biological
systems~\cite{barsky2008cerebral}, and social
networks~\cite{rothenberg1998using}. We compare our recreated
visualizations to the origianl layouts and discuss the benefits of our
technique for creating highly customized graph layouts.

\todo{Do we want to include disclaimers somewhere in this section that not
  all the recreations may be an exact match to those seen due to
  transcription errors or lack of information available in the original
  publication?}

\todo{At least one of the demonstrations we include should show that we can
  reuse a graph specification across graphs with similar properties.}

% running example in design section?

% comparison of user spec and generated spec (quantiative comparison)
% point out measure of comprehension
% our demo uses domain props whereas low-level constraints in _id (separate from original intent)

% simplicity and generalizability --> conciseness and expressiveness
%     (efficiency or learnability are other ideas)
% 2b spacing on nodes
% syphilis (does circle order matter?)


\subsection{Serengeti Food Web}
\krugerLayout
\serengetiLayout
\serengetiSpec

Food webs visualize complex producer-consumer relationships in ecological
systems and are a common presentation strategy for this information
\cite{hinke2004visualizing,harper2006dynamic,lavigne1996cod,baskerville2011spatial,yodzis1998local,cohen2003ecological,benson2016higher}
\orange{kruger citation} despite the challenge of creating an informative
visualization \cite{kearney2016blog}. Small or simplified food webs may be
drawn by hand, but many real world ecosystems can have hunderds of
interconnected organisms. In such cases, a customized layout may be useful
for reasoning about the structure of the ecological system.

Small food webs exhibit several of the properties of larger food webs, such
as a hierarchy arranged by the trophic level of the individuals in the
network. For example, the figure for Kruger National Park includes a simple visualization
of such relationships for a subset of the species found in the park
(Figure~\ref{fig:kruger-layout}a). We can easily recreate the layout with a
small number of constraints on the nodes (Figure~\ref{fig:kruger-layout}b);
in particular, we constrain each trophic level to be aligned and enforce an
ordering of the layers that respects the food web hierarchy. This
\projectname specification could easily be applied to other food webs to
produce a similar layout. \todo{this could be an easy place to have a
  second small food web with the same spec/layout?}. However, as the food
web gets more complex, it becomes necessary to relax the alignment
constraints or introduce additional clustering to highlight other
structures within the layout.

One example of a larger ecological network is the Serengeti food web from
Baskerville et al.~\cite{baskerville2011spatial}, which depicts the
relationships among 161 plants, herbivores, and carnivores with 592 links
between entities. Baskerville et al.\ use a Bayesian analysis method to
identify group structure for the Serengeti food web, and provide a
customized layout of their results showing both the trophic hierarchy and
the clustering of the groups (Figure \ref{fig:serengeti-layout}a). This
analysis approach and visualization highlight relationships between plant
habitats and the underlying network structure that may be hard to identify
from the data alone.

We recreated the Serengeti food web visualization in
\projectname~(Figure~\ref{fig:serengeti-layout}b), using the specification
shown in Figure \ref{fig:serengeti-spec}. \todo{include a description of
  what the specification is doing, what constraints we use, and the
  statistics for the final layout.} \todo{Maybe include something like
  Fig. 7 from the class paper?}

The original visualization was published in \textsc{Plos} Computational
Biology \cite{baskerville2011spatial}, and an interactive version of the
figure was created with D3 \cite{baskerville2011interactive}. The
interactive version of the figure lacks the labeling and alignment of the
published figure, but allows the viewer to more easily trace the links
between different nodes in the network. With \projectname, the layout
structure is specified by the constraints, which could support similar
interactive behaviors while maintaining the structural features of the
published figure.

Baskerville et al.\ note that \emph{``We have not included invertebrates
  (insects and parasitic helminths) or birds''} in their published food
web, though \emph{``hypothesize that the general conclusions will be
  largely robust to the addition of more species.''} One advantage of
\projectname~is that the layout is independant of the individual nodes, so
the authors could easily reuse this layout to visualize future iterations
of the Serengeti food web or to explore similar structures across different
ecological communities.

\subsection{TLR4 Network}
\todo{Get tlr4 image}

\todo{Write section about tlr4 network}

\subsection{Syphilis Social Network}
\syphilisLayout
\syphilisSpec

\todo{Jeff noted that the nodes in the circle do not have the same order -
  did they have an important order before worth capturing, if so, what?}

Social networks can be a powerful way to understand inter-personal
relationships and are useful for tracking the spread of diseases that
result from personal contact \cite{rothenberg1998using} \orange{[other
    citations]}. The ability to track and identify individuals at risk can
help lead to treatment and help manage the spread of the disease. In
addition to the links between individuals, structuring the graph layout by
properties such as ethnic group may reveal additional details about the
spread of the disease.

Rothenberg et al.\ discuss an ethnographic approach to identifying the
``core'' groups in a social network to better understand the transmission
of syphilis amongst sexual partners~\cite{rothenberg1998using}. They found
that there were three primary groups involved -- affluent white men (ages
17-21), young white women (ages ??-??), and African-American men (ages
16-??), which are visualized from left to right in Figure
\ref{fig:syphilis-layout}a. The authors note that a number of outsiders
(visualized as the top cluster of Figure \ref{fig:syphilis-layout}a) to
these ``core'' groups played a significant role in the network. In
particular, \emph{``Visualization of these groups and all their sex
  partners uncovered the importance of several people not specifically
  identified with these groups (e.g., N43 and S30) who served as bridges
  between the two groups of men.''}

\todo{I don't have the N43 bridge link in my visualization as that was not
  clear from any of the figures in their paper...}

We reproduced this example with
\projectname~(Figure~\ref{fig:syphilis-layout}b) and included a number of
additional constraints on the layout (Figure~\ref{fig:syphilis-spec}) apart
from the separation constraints that are visible in the original image. In
particular, we included a circle constraint on the group of young white
women to more strongly enforce the result shown in the original figure and
applied an alignment constraint on the two groups of young
men. \todo{discussion of the statistics for the final layout.}

\subsection{\orange{<<Fourth example>>}}
\todo{Another example with more reusability / different layouts}
