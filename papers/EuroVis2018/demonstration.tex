%!TEX root = constraint-layout.tex
\section{Examples Reproduced in \projectname}
\label{sec:examples}
To demonstrate the simplicity and generalizability of \projectname, we reproduce several real-world examples from ecological networks~\cite{baskerville2011spatial}, biological systems~\cite{barsky2008cerebral}, and social networks~\cite{rothenberg1998using}. We compare our recreated visualizations to the origianl layouts and discuss the benefits of our technique for creating highly customized graph layouts.

\todo{At least one of the demonstrations we include should show that we can reuse a graph specification across graphs with similar properties.}

% running example in design section?

% comparison of user spec and generated spec (quantiative comparison)
% point out measure of comprehension
	% our demo uses domain props whereas low-level constraints in _id (separate from original intent)


\subsection{Serengeti Food Web}
\serengetiLayout
\serengetiSpec
Food webs visualize complex producer-consumer relationships in ecological systems and are a common presentation strategy for this information \cite{hinke2004visualizing,harper2006dynamic,lavigne1996cod,baskerville2011spatial,yodzis1998local,cohen2003ecological,benson2016higher} despite the challenge of creating an informative visualization \cite{kearney2016blog}. Small or simplified food webs may be drawn by hand, but many real world ecosystems can have hunderds of interconnected organisms. In such cases, a customized layout may be useful for reasoning about the structure of the ecological system.

One example of a larger ecological network is the Serengeti food web from Baskerville et al. \cite{baskerville2011spatial}; this food web depicts the relationship between 161 plants, herbivores, and carnivores with 592 links between entities. Baskerville et al. use a Bayesian analysis method to identify group structure for the Serengeti food web and provide a customized layout of their results showing both the trophic hierarchy and the clustering of the groups (Figure \ref{fig:serengeti-layout}a). This analysis approach and visualization highlight relationships between plant habitats and the underlying network structure that may be hard to identify from the data alone.

We recreated the Serengeti food web visualization in \projectname~(Figure~\ref{fig:serengeti-layout}b), using the specification shown in Figure \ref{fig:serengeti-spec}. \todo{include a description of what the specification is doing, what constraints we use, and the statistics for the final layout.} \todo{Maybe include something like Fig. 7 from the class paper?}

The original visualization was published in \textsc{Plos} Computational Biology \cite{baskerville2011spatial}, and an interactive version of the figure was created with D3 \cite{baskerville2011interactive}. The interactive version of the figure lacks the labeling and alignment of the published figure, but allows the viewer to more easily trace the links between different nodes in the network. With \projectname, the layout structure is specified by the constraints, which could support similar interactive behaviors while maintaining the structural features of the published figure.

Baskerville et al. note that \emph{``We have not included invertebrates (insects and parasitic helminths) or birds''} in their published food web, though \emph{``hypothesize that the general conclusions will be largely robust to the addition of more species.''} One advantage of \projectname~is that the layout is independant of the individual nodes, so the authors could easily reuse this layout to visualize future iterations of the Serengeti food web or to explore similar structures across different ecological communities.

\subsection{TLR4 Network}
\todo{Get tlr4 image}

\todo{Write section about tlr4 network}

\subsection{Syphilis Social Network}
\syphilisLayout
\syphilisSpec
\todo{Get syhilis network image}

\todo{Write section about syphilis network}

\subsection{\orange{<<Fourth example>>}}
\todo{Another example with more reusability / different layouts}