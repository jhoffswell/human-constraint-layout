%!TEX root = constraint-layout.tex
\section{Limitations \& Future Work}

% constraints not satisfied and debugging and not have constraints respected by underlying solver

\todo{3. Talk about the need to pre-clean and post-edit the graphs. In
  particular, discuss the fact that this process is assuming that the
  graphs contain all the information we need as attributes. As such, we can
  defer some of the complexity that might come out of more complex
  selection techniques. Also discuss that we added some additional labeling
  to them in order to create the final 'figure'. But we did \emph{not} edit
  the layout.}

\todo{from Alan: is a benefit of deferring this complexity that it better
  supports prototyping and evolution of the visualizations, letting the
  user come up with a rough visualization quickly and then refining it? If
  so point this out.}

\todo{4. Talk about decision to allow nodes to occur in multiple sets in
  the creation process (which may result in the user specifying constraints
  that lead to contradictions.)}

% wrangler/polestar/lyra strategies for specifying constraints
% dsl leveraged as useful interface for processing logic

\subsection{Unsatisfiable Constraints in \projectname}
For a \projectname\ specification, it is possible to create node sets that
are \emph{not} disjoint and may thus be easily susceptible to the
specification of unsatisfiable constraints. Furthermore, through various
combinations of set specifications and constraint applications, it is
possible for the user to specify unsatisfiable constraints in the
\projectname\ layout. An example of a layout with unsatisfiable constraints
is shown in Figure \ref{fig:contradiction-example}. This layout completes
two partitions of the nodes (once by \texttt{level} and once by
\texttt{rank}); these collections lifted and an order constraint is applied
to enforce a hierarchy between the sets. However, due to the properties of
the nodes, the node with \texttt{rank 10} and \texttt{level 3} faces a
contradiction in the layout. Figure \ref{fig:contradiction-example}c shows
the error on the nodes that have constraints unsatisfied by the layout
(where darker means more error).

\jheer{This section notes that conflicts can arise, but not much else. What facilities if any does the language provide to help with this? How might a user learn that they've authored conflicting constraints? If this is more of a language runtime issue, we might note that there. For example at this point we haven't said anything about how our high-level constraints get translated into low-level constraints.}

\subsection{Limitations of WebCoLa as the Constraint Solver}

% remember to be generous with WebCoLa when discussing limitations; using a
% system like this makes it easier to uncover bugs or identify
% surprising/unsual behavior in the underlying system.

% what happens to the setCoLa specification when it is NOT on top of
% another system? (e.g. this is applying constraints on top of a number of
% other parameters of the underlying layout solver, so what is it's role
% alone?)

% configuration option for the underlying behavior? e.g. stress
% minimization, edge-length optimization, etc.

Our implementation with WebCoLa allows us to demonstrate the utility of
\projectname~for creating customized, generalizable layouts. However, our
implementation was constrained by what WebCoLa currently supports. For
example, we were unable to express \projectname's circle constraints in
WebCoLa as it does not currently include such a constraint type, though it
has been identified in the wiki as an area of future work
\orange{citation?}. This work contributes new strategies for the
specification of graph layout constraints, but does not aim to create a
highly optimized constraint solver for the language; Future work should
explore the design of a constraint solver optimized for \projectname~and
should look at how new or existing algorithms might co-evolve alongside the
high-level language.

\todo{Talk about fact that circle constraint is not particularly well
  defined in our implementation, but approximates the layout in webcola as
  an example. or just exclude it from the implementation as something we
  cannot express?}

We also encountered some behavioral mismatches between the implementation
of WebCoLa and our expectations for the graph layout. For example, WebCoLa
utilizes a default link length for the layout which attempts to optimize
node positions to produce links as close to the desired link length as
possible. This technique is a common strategy for force-directed layouts
\orange{citation} and helps to highlight the underlying structure of the
graph. For example, in the syphilis social network
(Figure~\ref{fig:syphilis-layout}b), since more links exist between the
young women and the african american men, the circle is drawn slightly off
center between the groups, which emphasizes the strength of these
connections. However, some layouts require node positioning that should be
independent of the graph structure. Revisting the syphilis social network
(Figure~\ref{fig:syphilis-layout}b), many of the nodes in the upper group
are drawn to the lower position allowed by the boundary guide in an attempt
to shorten the link length, thus producing an alignment unspecified in the
layout. \todo{include additional conclusions about this limitation, or
  discuss our results further?}

\todo{Talk about non-overlap problem for node layout (e.g., it is applied
  at the start of the layout and it can make it hard for nodes to follow
  the layout as they cannot pass over each other)}

\todo{In addition to the non-overlap problem, this becomes a problem since
  we use 'temporary' nodes as boundaries and guides in the layout. These 
  nodes have a weight of their own which can cause the layout to get stuck
  when the nodes don't overlap. I'm 'fixing' this by adding some initial
  randomness to the placement of the guide nodes so that they don't
  interfere with eachother}
