%!TEX root = constraint-layout.tex
\section{Introduction}
Graph visualizations can effectively represent properties of the underlying data structure, such as the hierarchy or connectedness of the data within the graph layout. Such visualizations are common across domains including ecological networks \cite{hinke2004visualizing,harper2006dynamic,lavigne1996cod,baskerville2011spatial,yodzis1998local,cohen2003ecological,kearney2016blog,benson2016higher}, biological systems \cite{barsky2008cerebral,shannon2003cytoscape,gehlenborg2010visualization,saraiya2005visualizing,becker2001graph}, and \orange{third example from demonstration}, among many others. In order to emphasize relevant trends in the data, the graph layout may utilize domain specific details; for example, in an ecological network, nodes could be split by trophic level (plant, herbivore, or carnivore) (Figure~\ref{fig:serengeti-layout}). This customized layout can more effectively reveal the feeding relationships between trophic levels and the hierarchical structure.

% Provide a bit more info about what the layout tools are doing - an additional setnece/rewording
% Not sure about 'layout tool' - cerebral is a midpoint between algorithm and customized
\todo{extend discussion of what the layout tools are doing (and tools might not be the right word/framing, algorithm seems better)}
Many domain specific layout tools have been developed to address particular graphing needs \cite{barsky2008cerebral,shannon2003cytoscape,kearney2017d3,kearney2017ecopath}. However, when a layout tool does not exist for the domain or task of interest, users are required to either fit their data to one of the available tools or create a customized layout of their own. Creating such a customized layout often requires both domain and programming expertise. \todo{unclear who is collaborating at this point.}
% Unclear who is collaborating at this point; a visualization expert?
When collaboration is infeasible, domain experts may be required to invest large amounts of time into learning the skills necessary to program the customized layout themselves. Furthermore, these customized tools are hard to generalize beyond their domain; these tools are carefully crafted to meet the needs of a particular domain or task, which means that it can be hard for domain experts to share their new programming expertise with others. This barrier to creating highly customized, domain specific layouts means there is a gap between the analysis needs of some domains and the availability of tools to handle those needs.

\todo{talk about constraint based solutions and the shortcomings before introducing our approach}

% See more about constraint based solutions and what are the shortcomings
% talk about tedium of cosntraint side - and the constraint based approaches

\todo{Need to come up with a careful, descriptive definition of what I mean by customized layouts at some point in the intro}
To enable the design of customized domain specific layouts with minimal programming expertise, we present a high-level constraint language for graph layout. Users partition nodes into sets based on node or graph properties and apply layout constraints within or between these sets. This approach allows users to specify layout requirements at a high-level, while deferring the computation of the node-level constraints to the underlying system. These constraint specifications reduce programming effort while still enabling highly customized and generalizable graph layouts. We demonstrate the utility of this technique with an implementation of our constraint language that compiles our high-level constraints into inter-node constraints for WebCoLa~\cite{WebCoLa}, a JavaScript library for constraint based graph layout. To demonstrate the ease and extensibility of our language, we recreate a number of customized layouts with our WebCoLa implementation. We show that users can compactly specify complex graph layouts that resemble those produced by customized layout engines.