%!TEX root = constraint-layout.tex
\section{Introduction}
With an appropriate graph layout, node-link diagrams can
can effectively convey properties of an underlying
network structure, such as hierarchy or connectedness. Such visualizations are common across domains including ecological networks
\cite{hinke2004visualizing,harper2006dynamic,lavigne1996cod,baskerville2011spatial,yodzis1998local,cohen2003ecological,kearney2016blog,benson2016higher,kruger2017},
biological systems
\cite{barsky2008cerebral,shannon2003cytoscape,gehlenborg2010visualization,saraiya2005visualizing,becker2001graph},
and social networks
\cite{scott1988social,rothenberg1998using,fitzpatrick2001preventable,mcelroy2003network,fu2011hiv}, 
among many others. A graph layout may utilize
domain-specific details in order to emphasize relevant patterns in the
data. 
%In an ecological network for example, nodes could be split by trophic
%level (e.g., its positition in the food chain) to highlight the hierarchy
%of producer/consumer relationships
%(Figure~\ref{fig:kruger-layout}). This visualization shows a customized
%layout: a unique layout that leverages knowledge of the data domain for
%node placement and understanding of the graph structure.
In a biological pathway, nodes could be layered by their 
subcellular location to provide an overview of the cellular structure
(Figure~\ref{fig:tlr4-layout}). The additional ``downstream genes'' layer
can further show the outcomes of this network, clustered by molecular function.

\tlrfourLayout

Many domain-specific layout techniques address particular needs
for customized layouts~\cite{barsky2008cerebral,genc2003constrained,shannon2003cytoscape,kearney2017d3,kearney2017ecopath}. These
techniques leverage common structural details that are significant to the
domain, such as known data hierarchies including cellular 
structure (Figure~\ref{fig:tlr4-layout}) or trophic level
(Figure~\ref{fig:kruger-layout},~\ref{fig:serengeti-layout}), as a guiding property of the
layout. However, these techniques rarely generalize beyond the domain for
which they were designed. There thus exists a mismatch between domains
where customized layouts would be useful and the availability of tools to
support those domains. When a layout technique does not exist for the
domain of interest, users are required to either fit their data
to available techniques or devise a new algorithm. Creating a
customized layout algorithm requires both
domain and programming expertise, and so introduces a gap between 
analysis needs and the availability of techniques to
handle those needs.

Constraints can be used to specify desired properties for a layout, as the designer can think about how domain-specific properties should control the positions of nodes. However, many existing techniques
for constraint layout require the user to define constraints
on individual nodes or node pairs in the graph. This process can be labor
intensive, requiring thousands of similar constraints and careful reasoning
about which nodes should be constrained. Moreover, instance-level constraints (e.g., defined \emph{extensionally} via node indices) prevent
reuse of a layout for various graphs from the same domain.

To enable customized domain-specific layouts with reduced
programming effort, we contribute \projectname: a domain specific language for
specifying high-level constraints for graph layout. Users partition nodes
into sets based on node or graph properties, and apply layout constraints
to these sets. This approach allows users to specify layout
requirements at a high-level, deferring the generation of
instance-level constraints to the underlying runtime system. These constraint
definitions reduce specification effort while enabling highly
customized and reusbale graph layouts.

Our implementation of \projectname generates instance-level constraints for
WebCoLa~\cite{WebCoLa}, a JavaScript library for constraint based graph
layout. We compile \projectname constraints to inter-node WebCoLa constraints.
To demonstrate the expressiveness of our language, we
recreate a number of customized layouts found in the scientific research 
literature. We
show that users can compactly specify complex graph layouts that resemble
those produced by customized layout engines, and can reapply layout
specifications across different graphs.
