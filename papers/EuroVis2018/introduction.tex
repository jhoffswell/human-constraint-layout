%!TEX root = constraint-layout.tex
\section{Introduction}
Graph visualizations can effectively represent properties of the underlying
data structure, such as the hierarchy or connectedness of the data within
the graph layout. Such visualizations are common across domains including
ecological networks
\cite{hinke2004visualizing,harper2006dynamic,lavigne1996cod,baskerville2011spatial,yodzis1998local,cohen2003ecological,kearney2016blog,benson2016higher,kruger2017},
biological systems
\cite{barsky2008cerebral,shannon2003cytoscape,gehlenborg2010visualization,saraiya2005visualizing,becker2001graph},
and social networks \cite{rothenberg1998using} \orange{[additional
    citations]}, among many others. The graph layout may utilize
domain-specific details in order to emphasize relevant trends in the
data. In an ecological network for example, nodes could be split by trophic
level (e.g., its positition in the food chain) to highlight the hierarchy
of producer/consumer relationships
(Figure~\ref{fig:kruger-layout}). This visualization shows a customized
layout: a unique layout that leverages knowledge of the data domain for
node placement and understanding of the graph structure.

Many domain-specific layout techniques address particular graphing needs
for customized layouts~\cite{barsky2008cerebral,shannon2003cytoscape,kearney2017d3,kearney2017ecopath}. These
techniques leverage common structural details that are significant to the
domain, such as known data hierarchies including trophic level
(Figure~\ref{fig:kruger-layout}, \ref{fig:serengeti-layout}) or cellular 
structure (Figure~\ref{fig:tlr4-layout}), as a guiding property of the
layout. However, these techniques rarely generalize beyond the domain for
which they were designed. There thus exists a mismatch between domains
where customized layouts would be useful and the availability of tools to
support those domains. When a layout technique does not exist for the
domain or task of interest, users are required to either fit their data
to available techniques or create a new algorithm of their own. Creating a
customized layout algorithm can be \krugerLayout
challenging and often requires both
domain and programming expertise; many domain
experts may lack the skills to create custom layouts themselves, thus
forcing them to invest large amounts of time into learning the necessary
skills or creating a layout by hand. This barrier introduces a gap between
the analysis needs of some domains and the availability of techniques to
handle those needs.

Constraints are a natural way to identify the desired properties for a
customized layout as the designer can think about how domain-specific
properties drive the positions of nodes. However, many existing techniques
for constraint layout require the user to identify individual constraints
on nodes or node \emph{pairs}in the graph. This process can be highly labor
intensive, requiring thousands of similar constraints or careful reasoning
about which nodes should be constrained. Furthermore, these constraints often
refer to low-level node properties (such as the node index) and thus prevent
the user from reusing the layout across graphs in the same domain.

To enable the design of customized domain-specific layouts with minimal
programming effort, we present \projectname: a domain specific language for
specifying high-level constraints for graph layout. Users partition nodes
into sets based on node or graph properties and apply layout constraints
within or between these sets. This approach allows users to specify layout
requirements at a high-level, while deferring the computation of the
node-level constraints to the underlying system. These constraint
specifications reduce programming effort while still enabling highly
customized and generalizable graph layouts. We demonstrate the utility of
this technique with an implementation of our constraint language that
compiles our high-level constraints into inter-node constraints for
WebCoLa~\cite{WebCoLa}, a JavaScript library for constraint based graph
layout. To demonstrate the ease and extensibility of our language, we
recreate a number of customized layouts with our WebCoLa implementation. We
show that users can compactly specify complex graph layouts that resemble
those produced by customized layout engines and can reapply layout
specifications across different graphs.
